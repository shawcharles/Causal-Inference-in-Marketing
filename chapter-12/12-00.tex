\chapter{Machine Learning for Nuisance and Heterogeneity}
\label{ch:ml-nuisance}
\index{double machine learning|(}
\index{CATE|(}

\noindent\textbf{Learning objectives.} This chapter shows how machine learning helps estimate nuisance functions in panels while preserving valid inference through \index{Neyman orthogonality}orthogonality and \index{cross-fitting}cross-fitting.

You will learn to formalise Neyman-orthogonal scores and dependence-respecting cross-fitting for panels, including staggered adoption designs. We develop \index{double machine learning}double/debiased ML estimators for average effects, dynamic effects, and \index{dose-response}dose-response functions, with appropriate aggregation. You will apply \index{CATE}heterogeneous treatment effects methods for panels, including \index{causal forest}causal forests, grouped effects, and \index{policy learning}policy learning. Finally, you will learn to specify tuning, overlap, trimming, leakage avoidance, and diagnostics, along with clustered and small-sample inference.

For related topics see: event-study and DiD aggregation (Chapters~\ref{ch:event} and~\ref{ch:did}); factor and hybrid methods (Chapters~\ref{ch:factor} and~\ref{ch:generalized-sc}); continuous treatments (Chapter~\ref{ch:continuous}); design diagnostics and inference (Chapters~\ref{ch:design-diagnostics} and~\ref{ch:inference}); high-dimensional controls (Chapter~\ref{ch:high-dim}).
