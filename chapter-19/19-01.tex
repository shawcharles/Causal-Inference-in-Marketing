\section{Motivation and Scope}
Credible causal inference depends on data that are fit for purpose. The metrics provided by advertising platforms—impressions, clicks, conversions—are not the same as the econometric estimands we seek. They are often assignment-dependent, mechanically correlated with the treatment, and subject to constant change from platform policy updates. As we have argued throughout this book, good practice demands that we are transparent about how we get from raw data to our final estimand \citet{angrist2010credibility}.

Mismeasured treatments and outcomes do more than inflate standard errors. When treatment assignment is observed with error, estimates suffer attenuation bias, shrinking toward zero and understating true effects. When the measurement error correlates with unobserved confounders, bias can go in either direction. Worse, measurement error can violate identification assumptions outright: a mismeasured instrument may fail the exclusion restriction; a mismeasured outcome may break parallel trends if the error process differs across groups \citep{bound2001measurement}.

This chapter outlines data sources in marketing panels, identity linking and join strategies under privacy constraints, transformations and aggregation choices that preserve identification, alignment of platform metrics with econometric estimands, governance and reproducibility standards, and validation strategies. We situate practice within modern panel frameworks \citet{arkhangelsky2024causal}. We link to identification in Chapter~\ref{ch:frameworks}, diagnostics in Chapter~\ref{ch:design-diagnostics}, threats in Chapter~\ref{ch:threats}, and inference in Chapter~\ref{ch:inference}.
