\section{Running Examples: Motivating Vignettes}
\label{sec:running-examples}

Theory without application is empty. This section revisits the three examples from Chapter 1, now equipped with the precise language of estimands, assumptions, and identification strategies. These vignettes will recur throughout the book as we apply each method.

The three examples span different panel dimensions, treatment types, and identification challenges:

\begin{itemize}
\item \textbf{Loyalty programme} (thin panel: 500 stores $\times$ 12 quarters): Binary staggered treatment. Key challenges: parallel trends, spillovers across stores.
\item \textbf{TV advertising} (fat panel: 50 DMAs $\times$ 100 weeks): Continuous treatment intensity. Key challenges: endogeneity, carryover effects.
\item \textbf{Platform entry} (square panel: 50 cities $\times$ 24 months): Binary staggered treatment. Key challenges: selection on unobservables, competitive responses.
\end{itemize}

\subsection*{Loyalty Programme with Staggered Rollout}

A retail chain operates 500 stores observed over 12 quarters. The chain launches a loyalty programme in a staggered fashion: 100 stores in quarter 3, 200 stores in quarter 5, 150 stores in quarter 8, and 50 stores never receive the programme during the sample period. The data form a balanced panel with $N = 500$ and $T = 12$, a thin panel where the number of units substantially exceeds the number of periods. Outcomes $Y_{it}$ are quarterly sales per store. Treatment $W_{it}$ is binary, equal to one if store $i$ has an active loyalty programme in quarter $t$ and zero otherwise. Covariates $X_{it}$ include store characteristics (size, location demographics, competitive intensity) and time-varying controls (local unemployment rate, competitor actions, holiday indicators).

The primary estimand is the ATT: the average effect of the programme on sales in treated stores and quarters, relative to the counterfactual of no programme. Given the staggered adoption structure, cohort-time effects $\text{ATT}(g, t)$ and event-time effects $\theta_k$ are also of interest, particularly to understand whether the programme effect grows over time as customers accumulate points and develop habits. Heterogeneity is anticipated: the programme may work well in affluent, low-competition areas but have negligible effects in saturated urban markets.

Identification relies on parallel trends: in the absence of the programme, treated and control stores would have experienced similar trends in sales. This assumption is testable in part through pre-trend diagnostics that check whether stores adopting in quarter 3 exhibit parallel trends with never-treated stores in quarters 1 and 2. Spillovers complicate identification: customers may refer friends, creating positive spillovers, or customers may cross-shop from nearby non-programme stores, creating geographic externalities. These spillovers violate SUTVA and require explicit modelling (Chapter~\ref{ch:spillovers}). Dynamic effects mean that $Y_{it}$ may depend on the entire treatment path $\underline{w}^t$, not just $W_{it}$, so event-study specifications (Chapter~\ref{ch:event}) that estimate $\theta_k$ for $k = 0, 1, 2, \ldots$ are essential.

Appropriate methods: staggered difference-in-differences (Chapter~\ref{ch:did}), event studies (Chapter~\ref{ch:event}), spillover models (Chapter~\ref{ch:spillovers}), causal forests (Chapter~\ref{ch:ml-nuisance}).

\subsection*{Television Advertising Carryover Across Markets}

A consumer packaged goods brand tracks sales in 50 designated market areas (DMAs) over 100 weeks. The brand varies television advertising intensity (measured in gross rating points, GRPs) across markets and weeks according to strategic considerations --- higher spending during product launches, in markets with strong past performance, and in response to competitive activity. The data form a panel with $N = 50$ and $T = 100$, a fat panel where the time dimension substantially exceeds the cross-sectional dimension. Treatment $W_{it}$ is continuous (GRPs purchased), and outcomes $Y_{it}$ are weekly sales. Covariates include online search volume, social media mentions, competitor GRPs, local economic indicators, and seasonal dummies.

The primary estimand is the dose-response function: how do sales respond to changes in TV GRPs, accounting for carryover effects that allow past advertising to influence current sales? The long-run multiplier and half-life of advertising effects are substantively important, as they quantify whether TV advertising has persistent effects or whether the impact dissipates quickly. Heterogeneity across markets --- urban vs rural, high income vs low income, competitive vs monopolistic --- is also of interest.

Identification is challenging because TV spending is endogenous: the brand increases advertising when it anticipates high demand or when competitors are active. Parallel trends is less natural with continuous treatment, but conditional independence --- conditional on observables $X_{it}$ and unit/time fixed effects, TV spending is as good as random --- can justify causal inference if the controls are sufficiently rich. High-dimensional controls (Chapter~\ref{ch:high-dim}) using lasso or double machine learning (Chapter~\ref{ch:ml-nuisance}) can flexibly adjust for many potential confounders without overfitting. Alternatively, factor models (Chapters 8 and 9) can capture common demand shocks that affect all markets, allowing identification from deviations of individual markets' advertising from the common trend.

Dynamic effects are central. A distributed lag model (Chapter~\ref{ch:dynamics}) provides one approach to capturing carryover. Such a model might specify
\[
Y_{it} = \alpha_i + \beta_t + \sum_{s=0}^S \gamma_s W_{i,t-s} + X_{it}' \delta + \varepsilon_{it},
\]
where $\{\gamma_s\}$ traces the effect of GRPs purchased $s$ weeks ago on current sales. The cumulative effect $\sum_{s=0}^S \gamma_s$ quantifies the total impact over $S$ weeks, and the half-life is the lag $s^*$ where $\sum_{s=0}^{s^*} \gamma_s = 0.5 \sum_{s=0}^S \gamma_s$. Synthetic control methods (Chapter~\ref{ch:sc}) offer an alternative: for a focal market-week with unusually high GRPs, construct a synthetic market from control markets with low GRPs, matching on pre-treatment sales trends and covariates, and compare post-treatment sales.

Creative execution, media context, and timing all affect advertising response. Panel methods that aggregate over these dimensions may obscure important variation or conflate compositional shifts with true treatment effects.

Measurement issues are salient. Nielsen TV ratings are based on samples, scanner sales data have coverage gaps, and attribution of sales to TV advertising is confounded by online search and social media activity. Chapter~\ref{ch:data-measurement} discusses measurement challenges and how to bound the bias introduced by imperfect data. Inference (Chapter~\ref{ch:inference}) must account for serial correlation within markets and potential cross-market correlation, pointing to two-way clustering or, given the long time series, HAC standard errors.

\subsection*{Platform Market Entry and Competitive Dynamics}

A food delivery platform enters 30 cities over a two-year period, with entry times staggered based on market size, regulatory environment, and operational capacity. The platform observes monthly restaurant revenues in 50 cities (30 treated, 20 never-treated) over 24 months, yielding a panel with $N = 50$ and $T = 24$, a square panel with roughly balanced dimensions. Treatment $W_{it}$ is binary (platform present in city $i$ in month $t$). Outcomes $Y_{it}$ are aggregate restaurant revenues. Covariates include city characteristics (population, income, density, number of restaurants), time-varying shocks (local economic indicators, pandemic phases, policy changes), and competitor actions (incumbent platform presence, pricing strategies).

The primary estimand is the ATT: the effect of platform entry on restaurant revenues in treated cities and months, relative to the counterfactual of no entry. Heterogeneity is important: small independent restaurants may benefit more than chain restaurants with established delivery operations because the platform provides access to delivery infrastructure they previously lacked. General equilibrium effects -- category expansion, labour market adjustments, consumer behaviour changes -- mean that the effect on treated restaurants may differ from the aggregate effect on the restaurant sector.

Identification faces several challenges. Each city is unique, so finding comparable control cities is difficult. The platform enters larger, more attractive cities first, creating selection on observables and unobservables. Parallel trends may hold conditionally after adjusting for covariates, or a factor structure may be needed to accommodate common shocks (macroeconomic trends, pandemic phases) that affect all cities differentially. Competitive responses create spillovers: incumbent platforms adjust pricing and marketing in response to entry, partially offsetting the treatment effect. These spillovers can be modelled explicitly (Chapter~\ref{ch:spillovers}) or bounded using partial identification techniques.

Synthetic control methods (Chapter~\ref{ch:sc}) are a natural starting point. For each treated city, construct a synthetic control city from never-treated cities, weighting to match pre-entry restaurant revenues. The post-entry difference between actual and synthetic revenues estimates the causal effect for that city. Appropriate methods also include synthetic difference-in-differences (Chapter~\ref{ch:generalized-sc}) and factor models (Chapters~\ref{ch:factor}, \ref{ch:advanced-matrix}), which combine synthetic control logic with panel structure and latent common shocks.

Spillover models (Chapter~\ref{ch:spillovers}) can quantify the competitive response by comparing revenues for restaurants in treated cities that do not join the platform to revenues in control cities. General equilibrium effects require comparing the total change in the restaurant sector (treated and untreated restaurants) to assess whether entry creates new demand or merely redistributes existing demand. Heterogeneous effects by restaurant type can be estimated using interactions in regression models or by stratifying the synthetic control analysis.

Across all three examples, the same methodological choices recur. Which assumption justifies the comparison: parallel trends, conditional independence, or factor structure? Which estimand answers the business question: ATT for ROI, event-time effects for dynamics, or dose-response for continuous treatments? Which diagnostics probe the conclusions: pre-trends, placebo tests, or sensitivity analyses? The methods developed in the following chapters provide the tools. These examples show how to wield them.
