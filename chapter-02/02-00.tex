\chapter{Causal Frameworks and Panel Notation}
\label{ch:frameworks}

This chapter develops the causal framework and panel notation that underpin all subsequent analysis. We formalise potential outcomes notation for panel data, including dynamic treatment paths, and define core estimands such as average treatment effects and event-time effects that recur throughout the book. We distinguish common data configurations -- proper panels, grouped repeated cross-sections, and row-column exchangeable data -- and classify assignment mechanisms that drive identification choices. Finally, we review basic regression mechanics and inference issues specific to panel settings. These foundations equip you with the conceptual vocabulary and technical prerequisites for the methods developed in later chapters.

\noindent\textbf{Learning objectives.} You will learn how to formalise potential outcomes for panels, define core estimands (ATE, ATT, event-time effects), classify data structures and assignment mechanisms, understand when basic regression tools succeed or fail, and map your data to appropriate methods via the crosswalk framework.
