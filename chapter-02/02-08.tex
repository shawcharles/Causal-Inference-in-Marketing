\section{Notation and Assumptions Reference}
\label{sec:notation-reference}

This section collects the core notation and assumptions for quick reference. Return here when encountering unfamiliar symbols or when checking which assumptions underpin a particular method. The formal definitions appear in earlier sections; here we provide a consolidated summary.

\subsection*{Notation}

We index units (stores, customers, markets, products) by $i = 1, \ldots, N$ and periods (quarters, weeks, months, years) by $t = 1, \ldots, T$. The observed outcome for unit $i$ in period $t$ is denoted $Y_{it}$, while potential outcomes under treatment level $w$ or treatment path $\underline{w}^t = (w_1, \ldots, w_t)$ are written as $Y_{it}(w)$ or $Y_{it}(\underline{w}^t)$. For staggered adoption with absorbing treatment, $Y_{it}(g)$ denotes the potential outcome if unit $i$ first adopted at time $g$, and $Y_{it}(\infty)$ denotes the never-treated potential outcome. The observed treatment is $W_{it}$, which is binary ($W_{it} \in \{0, 1\}$) in most chapters but continuous ($W_{it} \in \mathbb{R}$) in Chapter~\ref{ch:continuous}.

For staggered adoption designs, $G_i \in \{1, \ldots, T\} \cup \{\infty\}$ denotes the adoption cohort---the first period in which unit $i$ is treated. Units never treated have $G_i = \infty$ by convention. Event time $k = t - G_i$ measures periods since adoption, where $k = 0$ is the first treated period, $k < 0$ are pre-treatment periods, and $k > 0$ are post-treatment periods. Event-time indicators $D_{it}^k = \mathbf{1}\{t - G_i = k\}$ form the building blocks of event-study regressions.

Unit fixed effects $\alpha_i$ capture time-invariant unit heterogeneity, while time fixed effects $\lambda_t$ capture common shocks or trends. The idiosyncratic error is denoted $\varepsilon_{it}$.

The core estimands are the average treatment effect $\text{ATE} = \mathbb{E}_{i,t}[Y_{it}(1) - Y_{it}(0)]$ and the average treatment effect on the treated $\text{ATT} = \mathbb{E}_{i,t}[Y_{it}(1) - Y_{it}(0) \mid W_{it}=1]$. For staggered adoption designs, we also define the cohort-time effect $\tau(g,t) = \mathbb{E}[Y_{it}(g) - Y_{it}(\infty) \mid G_i = g]$ for $t \geq g$, and the event-time effect $\theta_k = \mathbb{E}[Y_{i,G_i+k}(G_i) - Y_{i,G_i+k}(\infty) \mid G_i < \infty]$, which measures the average effect $k$ periods post-adoption.

\subsection*{Core Assumptions}

The assumptions introduced earlier in this chapter are summarised here for quick reference. See the indicated sections for formal statements.

\paragraph{No Anticipation (Assumption~\ref{assump:no-anticipation}, Section~\ref{sec:potential-outcomes-panels}).}
Potential outcomes in period $t$ do not depend on treatments assigned in periods $s > t$. This rules out forward-looking behaviour where units respond to expected future treatments.

\paragraph{SUTVA (Assumption~\ref{assump:sutva}, Section~\ref{sec:potential-outcomes-panels}).}
Potential outcomes for unit $i$ depend only on unit $i$'s own treatment path, not on other units' treatments (\textbf{no interference}). There is a single, well-defined version of treatment (\textbf{treatment version irrelevance}).

\paragraph{Parallel Trends (Assumption~\ref{assump:parallel-trends}, Section~\ref{sec:assignment-mechanisms}).}
For all cohorts $g, g'$ and periods $t < \min(g, g')$, expected changes in untreated potential outcomes are the same across cohorts:
\[
\mathbb{E}[Y_{it}(\infty) - Y_{i,t-1}(\infty) \mid G_i = g] = \mathbb{E}[Y_{it}(\infty) - Y_{i,t-1}(\infty) \mid G_i = g'].
\]

\paragraph{No Dynamic Effects (Assumption~\ref{assump:no-dynamics}, Section~\ref{sec:core-estimands}).}
Potential outcomes depend only on current treatment status: $Y_{it}(\underline{w}^t) = Y_{it}(w_t)$. This rules out carryover and feedback from past treatments.

\paragraph{Homogeneous Treatment Effects (Assumption~\ref{assump:homogeneous-effects}, Section~\ref{sec:regression-mechanics}).}
The treatment effect is constant across units and time: $\tau_{it} = \tau$ for all $i, t$. Required for TWFE to recover a meaningful average effect under staggered adoption.

\paragraph{Unconfoundedness (Assumption~\ref{assump:unconfoundedness-continuous}, Section~\ref{sec:assignment-mechanisms}).}
Conditional on covariates $X_{it}$, unit fixed effects $\alpha_i$, and time fixed effects $\lambda_t$, treatment is independent of potential outcomes:
\[
W_{it} \perp Y_{it}(w) \mid X_{it}, \alpha_i, \lambda_t \quad \text{for all } w \in \mathcal{W}.
\]

\paragraph{Factor Structure (Section~\ref{sec:assignment-mechanisms}).}
Untreated potential outcomes follow a low-rank structure:
\[
Y_{it}(\infty) = \sum_{r=1}^R \lambda_{ir} f_{tr} + \varepsilon_{it},
\]
where $f_{tr}$ are latent factors, $\lambda_{ir}$ are unit-specific loadings, and $R \ll \min(N, T)$.

\medskip
These assumptions are invoked selectively depending on the method and context. Parallel trends underpins difference-in-differences and event studies. Factor structure justifies interactive fixed effects and matrix completion methods. Unconfoundedness is central to selection-on-observables approaches with high-dimensional controls. SUTVA is often violated in marketing and requires spillover modelling. No anticipation is testable in event studies through pre-treatment leads. Homogeneous effects is required for TWFE but relaxed by modern heterogeneity-robust estimators.

\subsection*{Forward References}

The notation and assumptions formalised here recur throughout the book. The parallel trends assumption and staggered adoption structure underpin the difference-in-differences and event-study methods developed in Chapters~\ref{ch:did} and \ref{ch:event}. Synthetic control methods (Chapters~\ref{ch:sc} and \ref{ch:generalized-sc}) relax parallel trends in favour of pre-treatment matching or factor structures.

Factor structure assumptions are operationalised in Chapters~\ref{ch:factor} and \ref{ch:advanced-matrix} through principal components, EM algorithms, and nuclear norm regularisation. Dynamic extensions appear in Chapter~\ref{ch:dynamics}, while Chapter~\ref{ch:spillovers} relaxes SUTVA to accommodate interference. Machine learning methods (Chapters~\ref{ch:ml-nuisance} and \ref{ch:high-dim}) leverage conditional independence with flexible estimators. Finally, Chapter~\ref{ch:inference} provides tools for uncertainty quantification, and Chapter~\ref{ch:design-diagnostics} develops diagnostics to assess assumption plausibility and sensitivity to violations.

This chapter establishes that vocabulary. The subsequent chapters show what you can do with it.