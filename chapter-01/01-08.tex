\section{Why Marketing Panel Data is Different}
\label{sec:why-marketing}

A reasonable question for any reader is: \textit{Why do we need a specialised book on panel data for marketing?} Excellent textbooks exist for biostatistics, financial econometrics, and labour economics. Can the marketing analyst not simply adopt the tools of the clinical trial or the asset pricing model?

The answer is that marketing data differs structurally from standard medical and financial data in ways that often break standard estimators. While marketing shares some DNA with modern epidemiology---observational studies of contagion and policy interventions---it is distinct from the randomised clinical trials that dominate biostatistics. The same features that make marketing data rich---strategic behaviour, social connectivity, high-frequency measurement---introduce specific violations of traditional assumptions. Four structural differences define the challenge.

\subsection*{Targeting Bias: Endogeneity by Design}

In clinical trials, treatment is randomised. In labour economics, treatment such as a minimum wage increase is often exogenous to the individual worker. In marketing, treatment is strategic and endogenous. Firms actively target advertising, coupons, and sales calls to customers most likely to buy (retargeting) or most likely to churn (retention campaigns).

Standard estimators do not merely yield biased estimates in this setting; they often get the sign wrong. A naive analysis of a retargeting campaign will show that treated users purchase at vastly higher rates than controls. This is not the effect of the ad; it is the cause of the targeting. \cite{blake2015consumer} demonstrated this starkly: when eBay suspended branded search advertising, the naive model implied large positive returns, but the experiment revealed near-zero incremental effect because ads were reaching users who would have purchased anyway. Disentangling targeting policy from treatment effect requires methods that go beyond fixed-attribute controls: negative control outcomes, double machine learning (Chapter~\ref{ch:ml-nuisance}), and synthetic control methods that match on pre-treatment trajectories (Chapter~\ref{ch:sc}).

\subsection*{Interference as the Norm}

The stable unit treatment value assumption---that each unit's outcome depends only on its own treatment---is reasonable in clinical trials (my taking aspirin does not cure your headache). In marketing, SUTVA is routinely violated, much as in infectious disease epidemiology. Social interference arises when a customer's purchase increases friends' purchase probability through word-of-mouth or visibility \citep{aral2011creating}. Spatial interference occurs when a television ad in one designated market area spills into adjacent DMAs. Competitive interference emerges when one firm's price cut triggers rival responses, altering equilibrium outcomes for all players.

Standard panel methods assuming unit independence fail to identify true effects when interference is present. Marketing applications require exposure mapping designs that model spillover structure explicitly, cluster-randomised experiments that contain interference within groups, or spatial econometric methods that parameterise cross-unit dependence (Chapter~\ref{ch:spillovers}).

\subsection*{Large-$N$, Moderate-$T$ Panels with Sparsity}

Clinical datasets typically feature small $N$ and short $T$---a hundred patients observed three times. Financial datasets feature moderate $N$ and very long $T$---five hundred stocks observed daily for twenty years. Marketing data occupies a different region: massive $N$ and moderate $T$. A retailer may track ten million customers over fifty-two weeks. The data are also sparse: most customers do not purchase in most weeks.

This structure renders traditional time-series methods (GARCH, cointegration) and traditional biostatistical methods (mixed models) computationally infeasible or statistically inefficient. It is, however, ideal for matrix completion and interactive fixed effects models (Chapters~\ref{ch:factor} and~\ref{ch:advanced-matrix}). These methods exploit the low-rank structure of consumer behaviour---millions of customers sharing a small number of latent preference dimensions---to construct counterfactual predictions with precision unattainable through unit-by-unit modelling.

\subsection*{Complex Dynamics: Adstock and Wear-out}

Financial markets are often modelled as efficient, incorporating information instantaneously. Marketing effects exhibit complex lag structures. An ad seen today may influence a purchase next week---the adstock or carryover effect formalised by \cite{broadbent1984spending}. But seeing the same ad repeatedly may reduce its effectiveness---wear-out or saturation. Habit formation creates state dependence, where past purchases shift future purchase probability.

Disentangling incremental lift from carryover requires distributed lag models and event study designs (Chapters~\ref{ch:event} and~\ref{ch:dynamics}) tuned to the stock-and-flow nature of marketing goodwill.

\medskip

Table~\ref{tab:domain-comparison} summarises these structural distinctions. While we draw inspiration from epidemiology for interference and from finance for high-frequency dynamics, the combination of strategic targeting, network spillovers, and massive sparse panels makes marketing analytics a distinct methodological field.

\begin{table}[t]
\centering
\begin{tighttable}
\caption{Structural Differences Across Domains}
\label{tab:domain-comparison}
\begin{tabularx}{\textwidth}{lXXX}
\toprule
\textbf{Feature} & \textbf{Clinical / Biostats} & \textbf{Financial Econometrics} & \textbf{Marketing Analytics} \\
\midrule
\textbf{Primary Goal} & Efficacy / Safety & Risk / Forecasting & \textbf{Incrementality / ROAS} \\
\addlinespace
\textbf{Assignment} & Randomised (RCT) & Systemic / Exogenous & \textbf{Strategic (Targeting)} \\
\addlinespace
\textbf{Interference} & Rare (except Vaccines) & Market-wide Equilibrium & \textbf{High (Social / Spatial)} \\
\addlinespace
\textbf{Data Shape} & Small $N$, Short $T$ & Small $N$, Long $T$ & \textbf{Large $N$, Moderate $T$} \\
\addlinespace
\textbf{Sparsity} & Low (Complete records) & Low (Continuous trading) & \textbf{High (Infrequent purchase)} \\
\addlinespace
\textbf{Key Method} & Survival / Mixed Models & Time Series / GARCH & \textbf{Synthetic Control / Matrix Factorization} \\
\bottomrule
\end{tabularx}
\end{tighttable}
\end{table}