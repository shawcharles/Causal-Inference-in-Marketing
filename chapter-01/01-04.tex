\section{Motivating Examples: Three Marketing Challenges}
\label{sec:motivating-examples}

To ground the abstract discussion of panel methods in concrete marketing problems, we consider three hypothetical scenarios that motivate the methods developed in subsequent chapters. Each scenario illustrates a distinct set of challenges and maps to a different constellation of panel data tools. While stylised, these examples reflect real measurement challenges practitioners face.

\subsection*{Example 1: Evaluating a Loyalty Programme with Staggered Rollout}

Consider a hypothetical retail chain operating 500 stores across multiple regions with three years of quarterly sales data. To increase customer retention and lifetime value, the chain launches a loyalty programme offering points for purchases, redeemable for future discounts. Rather than rolling out the programme simultaneously, the firm adopts a phased approach: 100 stores in year one, an additional 200 in year two, 150 more in year three. The remaining 50 stores — primarily small, remote locations — never receive the programme, serving as controls.

The staggered rollout reflects operational constraints and strategic considerations. Early adopters tend to be larger stores in high-income areas where management anticipates strong programme uptake. Later adopters include stores in more competitive markets, where the firm hopes the programme will help defend market share. This non-random assignment creates selection bias: programme stores differ systematically from non-programme stores in ways that affect sales even absent the programme. Simply comparing sales in programme stores to sales in non-programme stores would conflate the programme effect with these pre-existing differences.

The evaluation faces additional complications. Spillovers loom large: customers who join may refer friends or family, creating positive peer effects, while customers from nearby non-programme stores may switch to programme stores to earn rewards. The programme effect is unlikely to be immediate or constant — initial enrolment may be slow, habit formation takes time, and switching costs build gradually as customers accumulate points. There may even be anticipation effects if rumours of the impending launch spread beforehand. The effect likely varies across store types, with affluent, low-competition neighbourhoods seeing different responses than saturated urban markets where customers already have multiple loyalty options.

This scenario maps naturally to modern panel data methods. The staggered rollout calls for difference-in-differences estimators that accommodate heterogeneous adoption timing (Chapter \ref{ch:did}). Event-study specifications (Chapter \ref{ch:event}) plot treatment effects over time and enable pre-trend tests. Spillover models (Chapter \ref{ch:spillovers}) estimate geographic externalities. Causal forests (Chapter \ref{ch:ml-nuisance}) uncover heterogeneity across store types. Chapters \ref{ch:design-diagnostics} and \ref{ch:inference} develop the diagnostic workflow including placebo tests, leave-one-out analyses, and sensitivity analyses.

What might such an analysis reveal? In this hypothetical case, suppose the estimated average treatment effect were eight per cent across all programme stores and quarters. This aggregate effect would mask important dynamics: an initial effect of just two per cent in the first quarter post-launch, growing to eight per cent after four quarters and stabilising thereafter. Spillovers might be positive but modest — sales in non-programme stores within five kilometres of a programme store rising by two per cent, suggesting word-of-mouth effects. Heterogeneity analysis might reveal effects concentrated in high-income, low-competition areas (twelve per cent increase) with near-zero effects in saturated urban markets. These richer insights would guide not just whether to expand the programme, but where to expand it and how to manage expectations about the timeline for seeing results.

\subsection*{Example 2: Measuring Television Advertising Carryover in the Digital Age}

Suppose a consumer packaged goods brand seeks to understand the causal effect of TV advertising on sales, accounting for carryover effects and digital channel interactions. The hypothetical data consist of weekly observations for 50 designated market areas over 100 weeks, capturing gross rating points (GRPs), online search volume, social media mentions, and sales. The advertising agency varies TV spending strategically — higher during product launches, in markets with active competitors, and when prior sales trends suggest rising demand.

This endogeneity of TV spending is the first challenge. Markets that receive heavy TV advertising in a given week may be fundamentally different in that week from markets that receive little advertising. Even if we control for time-invariant market characteristics with fixed effects, time-varying confounders -- competitor actions, local economic shocks, seasonal patterns -- may drive both the advertising decision and sales, creating spurious correlation. A second challenge is carryover: TV advertising effects do not instantaneously materialise and vanish. Some viewers respond immediately by searching online or visiting a store. Others store the information and act days or weeks later. Brand awareness accumulates over repeated exposures and decays gradually in the absence of advertising. Specifying the functional form of this carryover is non-trivial and has been the subject of extensive research in the marketing mix modelling tradition. Third, cross-channel effects complicate the picture. TV advertising may increase online search, which in turn increases sales. If we estimate the total effect of TV on sales, we capture both the direct effect and the indirect effect mediated through search. But if we control for search in a regression, we may block the causal pathway and underestimate the TV effect. Understanding these mediated effects is substantively important but econometrically delicate. Fourth, measurement issues abound: Nielsen television ratings are based on panels that may not perfectly represent the full population, sales data are aggregated from retail scanner panels with their own coverage gaps, and seasonal effects (holidays, weather, major events) create non-stationarity that must be disentangled from treatment effects.

Several panel methods suit this scenario. Synthetic control methods (Chapters \ref{ch:sc}, \ref{ch:generalized-sc}) construct bespoke control groups for treated markets. Distributed lag models (Chapter \ref{ch:dynamics}) parameterise carryover structure. High-dimensional control methods (Chapter \ref{ch:high-dim}) handle many potential confounders through data-driven variable selection. Chapters \ref{ch:inference} and \ref{ch:design-diagnostics} cover cluster-robust inference, placebo tests, and sensitivity analyses for carryover assumptions.

In a hypothetical analysis, TV advertising might increase sales by five per cent in the campaign week, with a three-week half-life. Online search could mediate roughly forty per cent of the total effect, with TV driving search and search driving sales. Competitor advertising might partially offset the own-brand effect — reducing it by twenty per cent when competitors simultaneously increase spending. Carryover could prove stronger for emotional appeals than informational appeals, consistent with theories of memory and persuasion. Such insights would inform budget allocation, creative strategy, and competitive response.

\subsection*{Example 3: Platform Market Entry and Competitive Dynamics}

Consider a hypothetical food delivery platform (stylised examples include DoorDash, Deliveroo, or Uber Eats) entering 30 new cities over two years. The firm observes monthly restaurant revenues in both entry cities and comparison cities without platform entry. The data span 50 cities and 36 months, creating a panel with staggered treatment timing. The firm seeks to estimate the causal effect of platform entry on restaurant revenues, accounting for competitive dynamics and general equilibrium effects.

Several challenges complicate the analysis. Each city is unique, so finding exact matches for treated cities among the control cities is impossible. The firm selects entry cities based on market size, demographics, competitive landscape, and regulations. This introduces selection bias. Entry occurs at different times for different cities, with larger, more attractive markets entered first. Once the platform enters a city, incumbent platforms (competitors already operating in that city) may respond by lowering commissions, increasing marketing, or improving service quality, mitigating the treatment effect and inducing competitive spillovers. Moreover, platform entry affects not just the restaurants that join the platform but potentially the entire restaurant ecosystem: consumers may dine out more frequently (category expansion), delivery drivers shift labour supply, and even restaurants that do not join the platform may see changes in foot traffic or delivery orders through third-party services.

This scenario motivates several panel methods. Synthetic control (Chapter \ref{ch:sc}) constructs control cities from weighted averages of never-treated cities, with inference via permutation tests. Synthetic difference-in-differences (Chapter \ref{ch:generalized-sc}) combines unit and time weights for staggered entry settings. Factor models (Chapter \ref{ch:factor}) handle unobserved common shocks affecting all cities. Spillover models (Chapter \ref{ch:spillovers}) quantify competitive responses. We develop these methods and their diagnostics in Chapters \ref{ch:did} through \ref{ch:spillovers}.

Hypothetical findings might show platform entry increasing restaurant revenues by fifteen per cent on average, with substantial heterogeneity. Small, independent restaurants could see twenty-five per cent gains from expanded reach, while chains with established delivery operations might see only five per cent gains. The competitive response from incumbent platforms could offset the effect by roughly thirty per cent in markets where incumbents lower commissions or increase promotions. General equilibrium effects might be positive — the category expanding as consumers order more frequently — suggesting platform entry creates value rather than merely redistributing it, with implications for regulatory policy and market structure.

These three hypothetical examples illustrate the range of marketing questions panel data methods can address — selection bias, dynamics, spillovers, heterogeneity, measurement error, competitive interactions. Each will recur in subsequent chapters as we develop the technical methods and diagnostic workflows for credible causal inference.
