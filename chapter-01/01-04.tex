\section{Motivating Examples: Three Marketing Challenges}
\label{sec:motivating-examples}

To ground the abstract discussion of panel methods in concrete marketing problems, we consider three scenarios that motivate the methods developed in subsequent chapters. Each illustrates a distinct set of challenges and maps to a different constellation of panel data tools. While stylised, these examples reflect real measurement challenges practitioners face.

\subsection*{Example 1: Evaluating a Loyalty Programme with Staggered Rollout}

Consider a retail chain with 500 stores across various regions and three years of quarterly sales data. To increase customer retention and lifetime value, the chain launches a loyalty programme that rewards purchases with points redeemable for future discounts. Rather than launching the programme simultaneously, the company takes a phased approach: 100 stores in the first year, 200 more in the second, and 150 more in the third. The remaining 50 stores, mostly small and remote, never receive the programme and serve as controls.

The staggered rollout reflects both operational constraints and strategic considerations. Early adopters are typically larger stores in high-income areas where management anticipates strong programme uptake. Later adopters include stores in more competitive markets, where the company hopes the programme will help defend market share. This non-random assignment creates selection bias: programme stores differ systematically from non-programme stores in ways that affect sales even absent the programme. Simply comparing sales across store types would conflate the programme effect with pre-existing differences.

 Additional complications arise during the evaluation.  Spillovers are significant: customers who join may refer friends or family, resulting in positive peer effects, whereas customers from nearby non-programme stores may switch to programme stores to earn rewards.  The programme's effect is unlikely to be immediate or consistent — initial enrolment may be slow, habit formation takes time, and switching costs rise gradually as customers accumulate points.  There may even be anticipation effects if rumours about the impending launch circulate ahead of time.  The effect is likely to vary by store type, with affluent, low-competition neighbourhoods responding differently than saturated urban markets where customers already have multiple loyalty options.

This scenario maps naturally to modern panel data methods. The staggered rollout calls for difference-in-differences estimators that accommodate heterogeneous adoption timing (Chapter \ref{ch:did}). Event-study specifications (Chapter \ref{ch:event}) plot treatment effects over time and enable pre-trend tests. Spillover models (Chapter \ref{ch:spillovers}) estimate geographic externalities. Causal forests (Chapter \ref{ch:ml-nuisance}) uncover heterogeneity across store types. Chapters \ref{ch:design-diagnostics} and \ref{ch:inference} develop the diagnostic workflow including placebo tests, leave-one-out analyses, and sensitivity analyses.

What might such an analysis reveal? Suppose the estimated average treatment effect were eight per cent across all programme stores and quarters. This aggregate effect would mask important dynamics: an initial effect of just two per cent in the first quarter post-launch, growing to eight per cent after four quarters and stabilising thereafter. Spillovers might be positive but modest — sales in non-programme stores within five kilometres of a programme store rising by two per cent, suggesting word-of-mouth effects. Heterogeneity analysis might reveal effects concentrated in high-income, low-competition areas (twelve per cent increase) with near-zero effects in saturated urban markets. These richer insights would guide not just whether to expand the programme, but where to expand it and how to manage expectations about the timeline for seeing results.

\subsection*{Example 2: Measuring Television Advertising Carryover in the Digital Age}

Consider a consumer packaged goods brand seeking to understand the causal effect of TV advertising on sales while accounting for carryover effects and digital channel interactions. The data consist of weekly observations for 50 designated market areas over 100 weeks, including gross rating points (GRPs), online search volume, social media mentions, and sales. The advertising agency strategically varies TV spending, with higher levels during product launches, in markets with active competitors, and when previous sales trends indicate rising demand.

The first challenge is endogeneity. Markets receiving heavy TV advertising in a given week differ systematically from markets receiving little. Even with fixed effects controlling for time-invariant market characteristics, time-varying confounders—competitor actions, local economic shocks, seasonal patterns—may drive both advertising decisions and sales, inducing spurious correlation.

The second challenge is carryover. TV advertising effects neither appear nor disappear instantly. Some viewers respond quickly, searching online or visiting stores within days. Others store the information and act weeks later. Brand awareness accumulates through repeated exposure and decays gradually in the absence of advertising. Specifying the functional form of this carryover—geometric decay, polynomial distributed lags, flexible nonparametric shapes—has occupied extensive research in the marketing mix modelling tradition.

Third, cross-channel effects complicate estimation. TV advertising may increase online searches, which in turn drive sales. Estimating the total effect of TV on sales includes both direct and search-mediated indirect effects. But controlling for search in a regression risks blocking the causal pathway and underestimating the TV effect. Understanding mediation is substantively important but econometrically challenging.

Fourth, measurement error pervades the data. Nielsen ratings derive from panels that may not represent the full population. Sales data aggregate from retail scanner panels with their own coverage gaps. Seasonal effects---holidays, weather, major events---create non-stationarity that must be distinguished from treatment effects.

Several panel methods address these challenges.  Synthetic control methods (Chapters \ref{ch:sc} and \ref{ch:generalized-sc}) create customised control groups for treated markets.  Distributed lag models (Chapter \ref{ch:dynamics}) define carryover structure.  High-dimensional control methods (Chapter \ref{ch:high-dim}) use data-driven variable selection to address multiple potential confounders.  The chapters \ref{ch:inference} and \ref{ch:design-diagnostics} discuss cluster-robust inference, placebo tests, and sensitivity analyses for carryover assumptions.

What might such an analysis reveal? Television advertising could boost sales by 5\% during the campaign week, with a three-week half-life.  Online search may mediate approximately 40\% of the total effect, with TV driving search and search driving sales.  Competitor advertising may partially offset the own-brand effect, reducing it by 20\% when competitors simultaneously increase spending.  According to memory and persuasion theories, emotional appeals may have a stronger carryover than informational appeals.  Such insights would help guide budget allocation, creative strategy, and competitive response.

\subsection*{Example 3: Platform Market Entry and Competitive Dynamics}

Consider a food delivery platform such as DoorDash or Deliveroo that expands into 30 new cities over two years.  The company tracks restaurant revenues on a monthly basis in both entry and comparison cities that do not use the platform.  The data covers 50 cities over 36 months, resulting in a panel with staggered treatment timing.  The firm wants to estimate the causal effect of platform entry on restaurant revenue while controlling for competitive dynamics and general equilibrium effects.

 Several challenges complicate the analysis.  Because each city is unique, exact matches between treated and control cities are impossible.  The firm chooses entry cities based on market size, demographics, competitive landscape, and regulatory requirements.  This creates selection bias.  Entry occurs at different times in different cities, with larger, more appealing markets entering first.  Once the platform enters a city, incumbent platforms (existing competitors) may respond by lowering commissions, increasing marketing, or improving service quality, thereby mitigating the treatment effect and inducing competitive spillovers.  Furthermore, platform entry has an impact not only on the restaurants that join the platform, but on the entire restaurant ecosystem: consumers may dine out more frequently (category expansion), delivery drivers may shift labour supply, and restaurants that do not join the platform may see changes in foot traffic or delivery orders via third-party services.

This scenario motivates several panel methods. Synthetic control (Chapter \ref{ch:sc}) constructs control cities from weighted averages of never-treated cities, with inference via permutation tests. Synthetic difference-in-differences (Chapter \ref{ch:generalized-sc}) combines unit and time weights for staggered entry settings. Factor models (Chapter \ref{ch:factor}) handle unobserved common shocks affecting all cities. Spillover models (Chapter \ref{ch:spillovers}) quantify competitive responses. We develop these methods and their diagnostics in Chapters \ref{ch:did} through \ref{ch:spillovers}.

What might such an analysis reveal? Platform entry could increase restaurant revenues by fifteen per cent on average, with substantial heterogeneity. Small, independent restaurants could see twenty-five per cent gains from expanded reach, while chains with established delivery operations might see only five per cent gains. The competitive response from incumbent platforms could offset the effect by roughly thirty per cent in markets where incumbents lower commissions or increase promotions. General equilibrium effects might be positive — the category expanding as consumers order more frequently — suggesting platform entry creates value rather than merely redistributing it, with implications for regulatory policy and market structure.

These three examples illustrate the range of marketing questions panel data methods can address: selection bias, dynamics, spillovers, heterogeneity, measurement error, competitive interactions. Each will recur in subsequent chapters as we develop the technical methods and diagnostic workflows for credible causal inference.
