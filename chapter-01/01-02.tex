\section{What Are Panel Data Methods? A Conceptual Framework}
\label{sec:conceptual-framework}

Panel data methods take advantage of a fundamental asymmetry: by observing the same units repeatedly over time, we can learn about causal effects by comparing changes in outcomes for treated versus control units.  This within-unit, over-time variation provides us with leverage that cross-sectional or time-series analysis cannot match.

Consider a cross-sectional comparison. We observe that stores participating in a loyalty programme have higher sales than non-participating stores.  This correlation may indicate a causal effect.  However, it may also reflect pre-existing differences between stores that use loyalty programmes and those that do not.  Even if we measure and control for observable store characteristics such as size, location, and demographics, unobservable factors such as management quality or local brand strength can throw off the comparison.

A pure time-series approach performs no better. If a single store's sales increase after launching a loyalty programme, we cannot tell whether the programme caused the increase or whether sales would have risen anyway due to seasonal trends, competitive changes, or macroeconomic conditions.  Without a contemporaneous control group, we cannot distinguish between the treatment effect and the counterfactual evolution that would have occurred in the absence of the treatment.

Panel data methods combine the strengths of both approaches. By observing multiple stores over time, we compare the change in sales for stores that implement a loyalty programme to the change for stores that do not implement the programme during the same period.  If the stores in the two groups are similar enough to have followed parallel trends in the absence of the programme, the difference in changes identifies the programme's causal effect.  This difference-in-differences logic, which compares differences across groups to changes over time, eliminates both time-invariant confounding and common time trends.

The comparison table below summarises how panel data methods sit within the broader landscape of causal inference approaches:

\begin{table}[htbp]
\begin{tighttable}
\centering
\caption{Comparison of Causal Inference Approaches for Marketing}
\label{tab:method-comparison}
\begin{tabularx}{\textwidth}{Y Y Y Y Y}
\toprule
Method & Units & Time & Identification & Marketing Example \\
\midrule
Cross-section & Many independent customers or markets & Single snapshot ($t=1$) & Rich covariates support unconfoundedness; overlap diagnostics critical & One-off survey linking satisfaction to churn propensity with demographic controls \\
Time series & Single unit observed frequently & Long horizon with $T \gg 1$ & Weak/strong stationarity, no structural breaks, lag structure well-specified & Brand-level sales decomposed into trend, seasonality, and promotion shocks for a flagship product \\
Panel & Many units tracked over time & Balanced or unbalanced panel with $N$ and $T$ large & Parallel trends, interactive fixed effects, staggered adoption diagnostics; short-$T$ vs long-$T$ regimes, dynamic panels & Chain-wide store sales across 40 quarters to evaluate phased loyalty rollouts \\
Multi-way (matrix/tensor) panel & Multiple cross-sectional indices (e.g., stores $\times$ products $\times$ regions) & Many periods with outcomes arranged as matrices or tensors & Low-rank factor structure, matrix/tensor completion, structural break and cross-section dependence tests & SKU-by-store weekly panel used to recover missing sales and estimate treatment effects via matrix completion \\
Spatial/interference panel & Many units linked by geography or networks & Many periods with possible staggered exposure & Exposure mappings, partial interference, spillover and saturation models & Geo experiment where treated cities can shift demand into neighbouring markets via travel or word-of-mouth \\
High-dimensional/ML panel & Many units with rich covariates & Many periods, often with complex dynamics & Unconfoundedness with high-dimensional controls; orthogonalised ML scores, regularisation and post-selection inference; large-$N$, large-$T$ clustered asymptotics & User-level ad impressions with behavioural history used in double machine learning to estimate incremental lift \\
RCT & Many treatment and control units & Randomised in time or across units & Designed random assignment, compliance, SUTVA, no spillovers & Geo-level A/B test of email creative with pre-registered outcomes and clustered inference \\
\bottomrule
\end{tabularx}
\end{tighttable}
\end{table}

We must be precise about what panel data means. Observing the same individual units over multiple time periods is what distinguishes it and gives it statistical power. This differs from repeated cross-sections, which draw different samples of a population at different points in time.  Repeated cross-sections enable us to monitor group averages.  Panel data enables us to account for unobserved individual characteristics.

This advantage places panel methods within a broader hierarchy of causal reasoning. \citet{pearl2009causality} describes three ascending levels. At the lowest rung, association, we observe correlations: advertising expenditure moves with sales.  One step up, intervention, we consider the consequences of actions: what would happen to sales if we increased advertising expenditure by 10\%?  At the summit, we consider alternative histories: what would sales have been in the previous quarter if we had run a different campaign?

 Marketing strategy questions often necessitate intervention or counterfactual reasoning.  We want to know not only what patterns exist in historical data, but also what happens if we change our strategy in the future.  Predictive models and dashboards operate at the association level, identifying patterns but not supporting causal claims.  Panel methods, which make explicit assumptions about the assignment mechanism and the structure of potential outcomes, allow for interventional reasoning.  Under more stringent assumptions -- no dynamic effects, no spillovers, and proper outcome model specification -- some panel methods attempt counterfactual reasoning by imputing missing potential outcomes for treated units in treated periods.

Moving up Pearl's hierarchy requires stronger assumptions. Association requires only that we observe variables and quantify relationships. Intervention requires assumptions about confounders, parallel trends, and whether a factor structure captures unobserved heterogeneity. Counterfactuals require even stronger assumptions about stability of the data-generating process under hypothetical changes. Panel methods make these assumptions explicit, testable where possible, and subject to sensitivity analysis---an approach we emphasise throughout the book.

We develop these conceptual foundations systematically in Chapters~\ref{ch:did}--\ref{ch:spillovers}, always grounding abstract identification arguments in the concrete marketing measurement challenges introduced in Section~\ref{sec:measurement-crisis}.
