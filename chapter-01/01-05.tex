\section{The Two Cultures of Marketing Analytics}
\label{sec:two-cultures}

In an influential paper, statistician Leo Breiman distinguished two cultures in statistical modelling.  The first, ''data modelling,'' creates explicit probabilistic models and estimates parameters based on strong functional form assumptions.  The second, ''algorithmic modelling,'' treats the underlying mechanism as unknown and focusses on predictive accuracy via flexible, data-adaptive algorithms.  Breiman contended that statistics had long overemphasised the first culture while neglecting the second, resulting in missed opportunities to leverage the power of machine learning for prediction.

 This dichotomy, while influential, has been criticised for being too stark.  As Andrew Gelman and others have argued, good statistical practice has always included elements from both cultures.  Data modellers routinely validate their models' predictive performance using cross-validation and other methods.  Algorithmic modellers are increasingly seeking interpretability through tools such as SHAP values and partial dependence plots.  The real conflict is not between prediction and understanding, but between rigid parametric assumptions and adaptable, data-driven methods.  Importantly, neither culture, as originally conceived, directly addresses the central challenge of marketing analytics: shifting from prediction or association to *causal* inference.

 Today's marketing analytics reflect this tension.  The rise of digital platforms has fuelled a surge in predictive modelling, including churn prediction, recommendation engines, and bid optimisation.  These models excel at forecasting results *within* the current system.  However, when organisations face strategic decisions like, "Should we launch this loyalty program?"  Will the price change increase long-term revenue?  - Prediction alone is insufficient.  We need to make causal inferences.  Panel data methods, based on quasi-experimental design but augmented with modern machine learning, provide a path forward.

 Modern panel data methods for causal inference are a synthesis, or possibly a third culture.  They use the flexibility of algorithmic modelling to deal with high-dimensional confounders and estimate heterogeneous treatment effects.  However, they maintain the discipline of data modelling by making explicit assumptions about the causal structure, such as parallel trends or unconfoundedness, and subjecting those assumptions to diagnostic tests.  Methods such as Double Machine Learning exemplify this synthesis.  Machine learning algorithms estimate nuisance functions with few parametric assumptions, whereas design-based identification strategies ensure the final causal estimate is reliable.  This approach prioritises causal validity over pure predictive accuracy, but it does so using tools from both cultures.

Our approach to reasoning through models is pragmatic.  We are not looking for a single, true model that perfectly captures the data-generation process; such a model does not exist, and even if it did, we would not be certain we had found it.  Instead, we adopt a workflow that incorporates four components.  We begin by articulating the identification assumptions (parallel trends, conditional independence, factor structure, no spillovers) required for a causal interpretation of our results.  We then use an estimator that implements the identification strategy, which could be difference-in-differences, synthetic control, interactive fixed effects, or a doubly robust machine learning method.  After obtaining estimates, we perform a battery of diagnostics, including pre-trend tests, placebo tests, balance checks, leave-one-out robustness, covariate balance, and specification curves, to determine whether the assumptions are plausible and the results are sensitive to model choices.  Finally, we use sensitivity analysis to determine how large a violation of the key assumptions would be required to overturn our findings, while acknowledging that assumptions are approximations rather than exact truths.  This workflow represents cultural synthesis: flexible tools used within a disciplined causal framework.
