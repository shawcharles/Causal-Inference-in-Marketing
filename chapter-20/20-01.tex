\section{Motivation and Scope}
\label{sec:outlook-motivation}
 
Methods for panel causal inference have advanced rapidly, yet key challenges remain, especially in marketing, where data are shaped by platforms, privacy rules, and real-time optimisation. In this chapter we sketch a research agenda to tackle these open problems. Our approach remains grounded in the design-based principles of this book, with a sharp focus on clear estimands, transparent assumptions, and the inherent tension between prediction and identification \citet{angrist2010credibility, pearl2009causality, breiman2001statistical}.

\subsection*{What This Book Has Covered}

The preceding chapters have developed a toolkit for panel causal inference in marketing. Chapters~\ref{ch:frameworks}–\ref{ch:threats} set out the foundations, introducing potential outcomes, identification, and the design-based paradigm. Chapters~\ref{ch:did}–\ref{ch:factor} then developed the core methods, including difference-in-differences, synthetic control, factor models, and their extensions. We extended these tools in Chapters~\ref{ch:did}, \ref{ch:continuous}, \ref{ch:ml-nuisance}, and \ref{ch:spillovers} to handle staggered adoption, continuous treatments, the integration of machine learning, and interference.

Chapter~\ref{ch:inference} addressed inference, including variance estimation, the bootstrap, randomisation inference, and sensitivity analysis. Chapter~\ref{ch:applications} translated these ideas into marketing-specific protocols for advertising, pricing, platforms, and loyalty programmes. Finally, Chapter~\ref{ch:data-measurement} discussed data engineering, covering measurement, platforms, and reproducibility.

These tools address many practical problems, but the frontier continues to advance. New challenges emerge as platforms evolve, privacy constraints tighten, and marketing systems become more adaptive.

\subsection*{What Remains Open}

This chapter addresses twelve areas where current methods fall short or where guidance is needed. First, in Section~\ref{sec:interference-scale}, we consider \textbf{interference at scale} and ask how to estimate causal effects when millions of units interact through platforms, networks, and markets, for example in auction-based advertising and recommendation systems. Second, in Section~\ref{sec:nonstationarity}, we examine \textbf{nonstationarity and regime change} and how to handle environments where the data-generating process shifts because of policy changes, competitive dynamics, or algorithmic updates. Third, in Section~\ref{sec:adaptive-experimentation}, we study \textbf{adaptive experimentation} and the challenges of conducting causal inference when treatment assignment adapts in real time to observed outcomes. Fourth, in Section~\ref{sec:method-selection-outlook}, we turn to \textbf{method selection} and the problem of choosing among competing methods when assumptions are uncertain and diagnostics are imperfect. Fifth, in Section~\ref{sec:robust-inference-outlook}, we discuss \textbf{distribution-free and assumption-lean inference} and how to scale such approaches to complex panel dependence structures. Sixth, in Section~\ref{sec:continuous-outlook}, we revisit \textbf{continuous treatments} and the estimation of dose-response curves when overlap is limited and functional forms are unknown.

Seventh, in Section~\ref{sec:partial-identification}, we consider \textbf{partial identification} and how to report informative bounds when point identification fails. Eighth, in Section~\ref{sec:ml-beyond-nuisance}, we explore \textbf{ML integration} and how to use foundation models and generative AI for causal inference without introducing leakage or confounding. Ninth, in Section~\ref{sec:privacy-measurement}, we address \textbf{privacy-preserving measurement} and how to conduct causal inference when user-level data are unavailable and analysis occurs in data clean rooms, as is increasingly common in digital advertising. Tenth, in Section~\ref{sec:reproducibility-standards}, we examine \textbf{reproducibility and standards} and how to establish benchmarks and reporting standards for applied causal inference. Eleventh, in Section~\ref{sec:practitioner-roadmap}, we outline a \textbf{practitioner roadmap}, a workflow that practitioners can follow to apply these methods responsibly in real organisations. Twelfth, in Section~\ref{sec:future-assumptions}, we discuss \textbf{assumptions for future practice} and the new assumptions that may be needed as marketing environments continue to evolve.

\subsection*{Audience and Intent}

This chapter is addressed to both researchers and practitioners. For researchers, we identify open problems that merit formal investigation—problems where new theory, methods, or computational tools are needed. For practitioners, we highlight areas where current best practices are uncertain and where caution is warranted.

Our intent is not to provide definitive solutions. Many of these problems remain unsolved. Instead, we aim to articulate the questions clearly and to suggest directions for progress. The field advances when researchers and practitioners engage with the hardest problems rather than avoiding them.
