\section{Chapter Summary and Visual Guide}
\label{sec:outlook-summary}

This chapter has surveyed the open problems in panel causal inference for marketing. The challenges are substantial: interference at scale, structural instability, adaptive experimentation, privacy constraints, and the difficulty of method selection when assumptions are uncertain. The figures and table below provide a visual summary of these challenges and the research agenda they imply.

\begin{tcolorbox}[colback=gray!5!white,colframe=gray!75!black,title=Box 20.1: Key Takeaways from This Chapter]
\textbf{What we know how to do:} Estimate treatment effects in panels when units are independent, environments are stable, treatments are binary, and data are accessible. The methods in Chapters~\ref{ch:did}--\ref{ch:continuous} address these settings with increasing sophistication.

\textbf{What remains open:} Interference in dense networks where clusters overlap. Structural instability where regimes shift without warning. Adaptive systems where assignment depends on outcomes. Privacy constraints where user-level data are unavailable. Method selection where assumptions are uncertain and diagnostics are imperfect.

\textbf{What practitioners should do now:} Use the methods that exist, but acknowledge their limitations. Pre-register analysis plans. Run diagnostics before estimates. Report bounds when identification is partial. Document assumptions and governance. Triangulate across methods. Be honest about uncertainty.

\textbf{What the field needs:} New theory for interference and adaptivity. Scalable algorithms for large networks. Benchmarks with known ground truth. Reporting standards with enforcement mechanisms. Collaboration between academia and industry to access data and validate methods.
\end{tcolorbox}

\begin{figure}[htbp]
\centering
\includegraphics[width=0.95\textwidth]{images/fig_outlook_interference.pdf}
\caption{Interference-at-scale schematic with overlapping clusters and exposure mappings}
\label{fig:outlook-interference}
\small\textit{Panel A shows non-overlapping clusters (current tools). Three separate clusters (blue, green, yellow) with clean boundaries. Nodes (dark blue circles) belong to single clusters. This structure allows low-dimensional exposure mappings and standard partial-interference frameworks. Panel B shows overlapping clusters (open problem). Three clusters with overlapping boundaries (dashed lines). Red nodes belong to multiple clusters simultaneously. Network edges show spillover channels. This structure creates high-dimensional exposure mappings where units have complex, overlapping interference neighbourhoods. Challenges include scalable exposure construction, identification with overlapping memberships, and inference under network-induced dependence. Current tools assume Panel A structure. Dense platform ecosystems and social networks exhibit Panel B complexity.}
\end{figure}

\begin{figure}[htbp]
\centering
\includegraphics[width=0.95\textwidth]{images/fig_outlook_regime.pdf}
\caption{Regime-change timeline with breakpoints overlaid on outcomes and exposures}
\label{fig:outlook-regime}
\small\textit{Two-panel time series (120 periods) illustrates structural instability challenges. Panel A shows outcomes with three regime breaks (red dashed lines at periods 30, 60, 90). Regime 1: linear trend. Regime 2: slower growth. Regime 3: nonlinear dynamics. Regime 4: return to trend. Green shaded regions mark stable windows suitable for analysis. Breaks represent platform policy changes, algorithm updates, or macro shocks. Panel B shows time-varying treatment effect (orange line). Effect drifts across regimes: constant in Regime 1, increasing in Regime 2, declining in Regime 3, stabilising in Regime 4. Purple dotted line shows adaptive parameter tracking that updates in real time. Challenge: current methods assume stability or discrete known breaks. Platforms evolve continuously with latent breaks. Open problems include real-time adaptation, break detection integrated with factor dynamics, and robustness to shifting supports.}
\end{figure}

\begin{figure}[htbp]
\centering
\includegraphics[width=0.95\textwidth]{images/fig_outlook_method_selection.pdf}
\caption{Method-selection decision map linking threats to design families and diagnostics}
\label{fig:outlook-method-selection}
\small\textit{Four-level decision flowchart guides method selection. Level 1 (top): Data features include staggered adoption (blue), few treated units (green), dense network (yellow), continuous dose (lavender). Level 2: Threats include parallel trends violations, pre-period fit quality, spillovers, and confounding (red boxes). Gray arrows link features to relevant threats. Level 3: Design families include DiD/Event Study (cyan), SC/SDID (green), Spillover Models (yellow), and GPS/DML (lavender). Each design box lists key diagnostics (bullets): pre-trends and balance for DiD, weights and RMSPE for SC, exposure and buffers for spillovers, overlap and cross-fitting for GPS/DML. Level 4: Inference choices include cluster-robust SE, bootstrap/permutation, and joint bands with multiplicity control (brown boxes). Bottom: Green output box shows design-faithful causal estimate with diagnostics. Multiple paths possible: triangulate across methods. Decision flow: data features → threats → design families → inference. Formal frameworks for this mapping remain an open problem.}
\end{figure}

\begin{table}[htbp]
\small
\setlength{\tabcolsep}{5pt}
\renewcommand{\arraystretch}{1.1}
\centering
\caption{Open problems, current tools, research gaps, and candidate evaluation metrics}
\label{tab:outlook-open-problems}
\begin{tabularx}{\textwidth}{Y Y Y Y}
\toprule
\textbf{Open problem} & \textbf{Current tools} & \textbf{Research gaps} & \textbf{Evaluation metrics} \\
\midrule
Interference at scale & Partial interference, buffers & Overlapping clusters, dense networks & Spillover detection power, scalability \\
Structural instability & Break tests, time-varying factors & Real-time adaptation, latent breaks & Post-break coverage, regime detection \\
Adaptive experiments & Switchbacks, phased rollouts & Adaptivity + interference, sequential testing & Inference validity, regret bounds \\
Method selection & Diagnostics, triangulation & Formal decision frameworks, benchmarks & Bias-variance trade-offs, robustness \\
Distribution-free inference & Permutation, conformal & Staggered adoption, few clusters & Finite-sample coverage, computational cost \\
Continuous + dynamics & GPS, DML, ad-stock & Anticipation, endogenous rules & Equilibrium validity, overlap \\
Partial ID & Sensitivity analysis, bounds & Defaults, joint violations & Bound width, informativeness \\
ML integration & Orthogonalisation, cross-fitting & Representation stability, leakage & Prediction-ID balance, robustness \\
Privacy + federated & Aggregation, differential privacy & Federated DiD/SC, diagnostics & Privacy-accuracy trade-off, auditability \\
\bottomrule
\end{tabularx}
\end{table}

\subsection*{Looking Forward}

The open problems in Table~\ref{tab:outlook-open-problems} will not be solved quickly. Progress will come incrementally, through theoretical advances, computational innovations, and---perhaps most importantly---collaborations between researchers and practitioners who can access the data where these problems are most acute.

We wrote this book to provide practitioners with the best available tools while being honest about their limitations. The methods work well in settings they were designed for: independent units, stable environments, binary treatments, accessible data. As marketing moves toward dense networks, adaptive platforms, and privacy-constrained measurement, we will need new methods. Until those methods arrive, we must use what we have thoughtfully, report results honestly, and acknowledge what remains unknown.

The credibility of causal inference in marketing depends not on having perfect methods---we never will---but on practitioners who understand the methods' assumptions, apply them with care, and communicate their limitations. That is the mindset this book has sought to cultivate.
