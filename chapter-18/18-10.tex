\section{Subscription and Paywall Effects}
\label{sec:paywall-effects}

For media companies, the paywall is a critical lever: stricter paywalls increase subscription revenue but reduce reach, advertising impressions, and brand awareness. Estimating the causal effect of paywall strictness on subscriptions---and the trade-off with advertising revenue---requires careful design because users who hit the paywall differ systematically from those who do not.

\begin{remark}[Paywall Effects in the Taxonomy]\label{rem:paywall-taxonomy}
In the taxonomy of Section~\ref{sec:taxonomy}, paywall optimisation involves both \emph{confounding} and \emph{dynamics}. On the identification axis, high-engagement users self-select into paywall exposure, creating positive selection bias. On the temporal axis, subscription is an inherently dynamic outcome: it creates recurring revenue over months or years, and early-tenure experiences shape long-run retention. The short-run conversion effect understates lifetime value; see Section~\ref{sec:clv-acquisition} for CLV perspectives on customer acquisition. Subscription also parallels loyalty programme membership (Section~\ref{sec:loyalty-valuation}): both involve self-selection, ongoing engagement, and habit formation.
\end{remark}

\subsection*{Estimand: Paywall Conversion and Revenue Trade-off}

We define the causal effects of paywall policy on user behaviour.

\begin{definition}[Paywall Conversion Effect]\label{def:paywall-conversion}
Let $W_i \in \{0, 1\}$ indicate whether user $i$ encounters the paywall (e.g., exceeds the free article limit). The conversion effect is:
\[
\tau_{\text{convert}} = \mathbb{E}[S_i(1) - S_i(0) | W_i = 1],
\]
where $S_i \in \{0, 1\}$ indicates subscription. This is the ATT: the effect of hitting the paywall on subscription probability for users who would hit it.
\end{definition}

This estimand captures immediate conversion. The full value of paywall-induced subscriptions includes retention and lifetime value: subscribers acquired via the paywall generate revenue in subsequent periods and may develop reading habits that persist. Reporting both conversion rate and projected CLV provides a more complete picture.

The paywall also affects engagement and advertising revenue:

\begin{definition}[Paywall Reach Effect]\label{def:paywall-reach}
Let $Y_i$ be total page views or ad impressions. The reach effect is:
\[
\tau_{\text{reach}} = \mathbb{E}[Y_i(\text{strict}) - Y_i(\text{lenient})],
\]
comparing outcomes under strict (e.g., 3 free articles) vs. lenient (e.g., 10 free articles) paywall policies.
\end{definition}

The business question is whether the subscription revenue gain from a stricter paywall exceeds the advertising revenue loss from reduced reach.

\subsection*{Identification Challenge: Selection into Paywall Exposure}

Users who hit the paywall are high-engagement users who differ from casual visitors. Naive comparison of subscribers vs. non-subscribers among paywall-exposed users confounds the paywall effect with user heterogeneity.

\begin{assumption}[Selection into Paywall Exposure]\label{assump:paywall-selection}
Let $X_i$ be user characteristics (visit frequency, content preferences, device). Users with higher $X_i$ are more likely to hit the paywall and more likely to subscribe regardless of paywall exposure:
\[
P(W_i = 1 | X_i) \text{ increasing in } X_i, \quad P(S_i(0) = 1 | X_i) \text{ increasing in } X_i.
\]
This positive selection biases naive estimates upward.
\end{assumption}

\subsection*{Identification Strategy 1: Article Limit RDD}

If the paywall triggers at a fixed article count, regression discontinuity compares users just above and below the threshold.

\begin{definition}[Article Limit RDD]\label{def:paywall-rdd}
Let $A_i$ be the number of articles read in a billing period. If the paywall triggers at $\bar{A}$ articles, the RDD estimate is:
\[
\hat{\tau}_{\text{RDD}} = \lim_{a \downarrow \bar{A}} \mathbb{E}[S_i | A_i = a] - \lim_{a \uparrow \bar{A}} \mathbb{E}[S_i | A_i = a].
\]
This estimates the local effect of paywall exposure on subscription for users at the margin.
\end{definition}

\begin{assumption}[Continuity of Potential Outcomes at Article Threshold]\label{assump:paywall-rdd}
Potential subscription outcomes are continuous functions of article count at the threshold:
\[
\lim_{a \downarrow \bar{A}} \mathbb{E}[S_i(0) | A_i = a] = \lim_{a \uparrow \bar{A}} \mathbb{E}[S_i(0) | A_i = a].
\]
Since paywall exposure jumps discontinuously at $\bar{A}$, any discontinuity in observed subscription rates is attributable to the paywall.
\end{assumption}

This assumption fails if users strategically stop reading to avoid the paywall, creating selective bunching below the threshold.

\begin{remark}[Rational Avoidance and RDD Validity]\label{rem:paywall-bunching}
Users who stop at $\bar{A} - 1$ articles to avoid the paywall are engaging in rational avoidance, not manipulation in the sense of fraud. However, this behaviour affects RDD validity:
\begin{enumerate}
\item \textbf{Missing ``would-be treated.''} Users who would have read $\bar{A}$ articles but stopped short are absent from the treatment group. The RDD identifies the effect for users who \emph{could not or would not} avoid the threshold.
\item \textbf{Selected population.} Those who cross the threshold despite awareness may be less price-sensitive or more committed readers. The local effect may not generalise to the broader population.
\item \textbf{Bounds interpretation.} If avoiders have lower subscription propensity (they valued free content more), the RDD estimate is an upper bound on the population average effect.
\end{enumerate}
The McCrary density test detects bunching, but modest bunching (e.g., $p = 0.08$) warrants caution rather than rejection. Report the RDD estimate with appropriate caveats about external validity.
\end{remark}

\subsection*{Identification Strategy 2: Randomised Paywall Experiments}

The gold standard is randomising paywall strictness across users or time periods.

\begin{definition}[Paywall A/B Test]\label{def:paywall-ab}
Randomly assign users to paywall conditions $W_i \in \{\text{strict}, \text{lenient}, \text{none}\}$. The ATE is:
\[
\tau_{\text{ATE}} = \mathbb{E}[S_i | W_i = \text{strict}] - \mathbb{E}[S_i | W_i = \text{lenient}].
\]
\end{definition}

Randomisation eliminates selection bias but may have limited external validity if the experiment population differs from the full user base (e.g., new users only).

\subsection*{Identification Strategy 3: Staggered Paywall Rollout}

If paywall strictness changes over time or across regions, staggered difference-in-differences applies.

\begin{assumption}[Parallel Trends for Paywall Rollout]\label{assump:paywall-did}
In the absence of the paywall policy change, subscription trends would be parallel across cohorts:
\[
\mathbb{E}[S_{it}(0) - S_{i,t-1}(0) | G_i = g] = \mathbb{E}[S_{it}(0) - S_{i,t-1}(0) | G_i = g'], \quad \forall g, g',
\]
where $G_i$ is the cohort (rollout wave) for user $i$.
\end{assumption}

Use heterogeneity-robust estimators (Chapter~\ref{ch:did}) to avoid negative weighting when effects vary across cohorts.

\subsection*{Estimation}

For RDD, use local polynomial regression with the Calonico-Cattaneo-Titiunik bandwidth selector. The running variable is article count; the outcome is subscription indicator.

For A/B tests, simple difference-in-means with robust standard errors suffices. For long-horizon outcomes (e.g., 12-month retention), account for attrition.

For staggered rollout, apply Callaway-Sant'Anna or Sun-Abraham estimators with user-level clustering.

\subsection*{Revenue Trade-off Analysis}

The optimal paywall balances subscription and advertising revenue.

\begin{definition}[Paywall Revenue Trade-off]\label{def:paywall-tradeoff}
Let $R^S$ be subscription revenue and $R^A$ be advertising revenue. The net effect of moving from lenient to strict paywall is:
\[
\Delta R = \underbrace{\tau_{\text{convert}} \times \text{Price} \times N_{\text{exposed}}}_{\text{Subscription gain}} - \underbrace{\tau_{\text{reach}} \times \text{CPM} \times N_{\text{users}}}_{\text{Advertising loss}},
\]
where $N_{\text{exposed}}$ is users hitting the paywall and $N_{\text{users}}$ is total users.
\end{definition}

Report both components separately, as they accrue to different business units and have different time horizons.

\begin{remark}[Habit Formation and Lock-In]\label{rem:subscription-habits}
Subscription creates lock-in and habit formation that extend beyond the initial conversion:
\begin{enumerate}
\item \textbf{Reading habits.} Subscribers who develop daily reading routines have higher retention than those who subscribe impulsively.
\item \textbf{Sunk cost effects.} Having paid for a subscription, users may consume more content to ``justify'' the cost, reinforcing engagement.
\item \textbf{Switching costs.} Subscribers accumulate saved articles, personalised recommendations, and familiarity with the interface, raising barriers to cancellation.
\end{enumerate}
These dynamics imply that the long-run effect of paywall-induced subscription may exceed the short-run conversion effect. Marginal subscribers acquired via aggressive paywalls may, however, have lower retention than subscribers acquired organically. Tracking cohort-level retention by acquisition source informs long-run paywall optimisation.
\end{remark}

\subsection*{Diagnostic Checklist}

\begin{tcolorbox}[colback=gray!5!white,colframe=gray!75!black,title=Box 18.13: Paywall Diagnostic Checklist]
\textbf{For Article Limit RDD:}
\begin{itemize}
    \item Manipulation: Test for bunching just below the article threshold (McCrary test).
    \item Threshold awareness: Survey or behavioural data on whether users know the limit.
    \item Covariate balance: Check that user characteristics are smooth through the threshold.
    \item Bandwidth sensitivity: Report estimates at 0.5×, 1×, and 2× optimal bandwidth.
\end{itemize}

\textbf{For A/B Tests:}
\begin{itemize}
    \item Balance: Verify randomisation succeeded (covariate balance across arms).
    \item Attrition: Check for differential dropout across paywall conditions.
    \item Spillovers: Users may share articles, contaminating the control group.
    \item Duration: Short experiments may miss long-run subscription effects.
\end{itemize}

\textbf{For Revenue Trade-off:}
\begin{itemize}
    \item Time horizon: Subscription value accrues over months; advertising loss is immediate.
    \item Heterogeneity: High-value users may respond differently to paywall than casual visitors.
    \item Cannibalisation: Stricter paywall may shift users to competitor sites.
\end{itemize}
\end{tcolorbox}

\subsection*{Case Study: News Publisher Paywall Optimisation}

We illustrate the RDD approach with a hypothetical news publisher. The numbers are illustrative and do not represent real data.

\paragraph{Setting.} A digital news publisher operates a metered paywall: users can read 5 free articles per month before encountering a subscription prompt. The subscription price is \$10/month. We estimate the effect of hitting the paywall on subscription conversion.

\paragraph{Data.} 2 million unique visitors over 6 months. Running variable is monthly article count; outcome is subscription within 30 days of hitting the paywall.

\paragraph{RDD Estimate.} At the 5-article threshold, subscription probability jumps from 0.8\% (just below) to 2.1\% (just above). The RDD estimate is $\hat{\tau} = 1.3$ percentage points (SE = 0.2 pp, $p < 0.01$).

\paragraph{Diagnostics.} The McCrary test shows modest bunching at 5 articles ($p = 0.08$), suggesting some users may stop reading to avoid the paywall. Covariate balance is acceptable: age, device type, and referral source are smooth through the threshold. The estimate is stable across bandwidths: 1.1 pp (half), 1.3 pp (optimal), 1.4 pp (double).

\paragraph{Revenue Trade-off.} Of 2 million visitors, 400,000 hit the paywall (20\%). The subscription gain is:
\[
0.013 \times \$10 \times 12 \text{ months} \times 400{,}000 = \$624{,}000 \text{ annual revenue}.
\]
However, users who hit the paywall and don't subscribe reduce their visits. Estimated reach loss is 15\% of page views among paywall-exposed non-subscribers, translating to approximately \$180,000 in annual advertising revenue at \$5 CPM.

\paragraph{Net Effect.} The net revenue gain from the 5-article paywall is approximately \$444,000 annually. A stricter 3-article paywall would increase conversions but also increase reach loss; the optimal threshold requires estimating the full trade-off curve.

\paragraph{Interpretation.} The paywall generates substantial subscription revenue that exceeds the advertising loss. However, the modest bunching suggests some users are gaming the system. The publisher might consider: (1) making the threshold less salient; (2) offering a "hard" paywall for high-value content; or (3) personalising the threshold based on predicted user value.
