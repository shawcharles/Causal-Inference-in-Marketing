\section{Duration Models for Takeoff in Panels}
\label{sec:duration-models}

Marketing innovations often have a distinct takeoff point when growth accelerates. Duration models offer a natural framework for analysing this time-to-takeoff, modelling the hazard of transition from pre-takeoff to post-takeoff states.

\begin{definition}[Hazard and Survival]\label{def:hazard-survival}
For duration $T_i > 0$ (e.g., time to takeoff), the survival function is:
\[
S(t) = P(T_i > t),
\]
the probability of surviving past time $t$. The hazard function is the instantaneous rate of failure:
\[
h(t) = \lim_{\Delta t \to 0} \frac{P(t \leq T_i < t + \Delta t \mid T_i \geq t)}{\Delta t} = \frac{f(t)}{S(t)},
\]
where $f(t) = -S'(t)$ is the density. The cumulative hazard is $H(t) = \int_0^t h(s) \, ds = -\log S(t)$.
\end{definition}

\begin{definition}[Cox Proportional Hazards]\label{def:cox-ph}
The Cox proportional hazards model specifies:
\[
h(t \mid X_i) = h_0(t) \exp(X_i'\beta),
\]
where $h_0(t)$ is the baseline hazard (unspecified) and $\exp(X_i'\beta)$ is the hazard ratio. For panel data with covariates varying over time:
\[
h(t \mid X_i(t)) = h_0(t) \exp(X_i(t)'\beta).
\]
The partial likelihood eliminates $h_0(t)$:
\[
\mathcal{L}(\beta) = \prod_{i: \text{event at } t_i} \frac{\exp(X_i'\beta)}{\sum_{j \in \mathcal{R}(t_i)} \exp(X_j'\beta)},
\]
where $\mathcal{R}(t)$ is the risk set at time $t$.
\end{definition}

\begin{definition}[Shared Frailty]\label{def:frailty}
The shared frailty model introduces unit-level random effects to capture unobserved heterogeneity:
\[
h(t \mid X_i, \nu_i) = h_0(t) \exp(X_i'\beta + \nu_i),
\]
where $\nu_i \sim \mathcal{N}(0, \sigma^2_\nu)$ or $\exp(\nu_i) \sim \Gamma(1/\theta, 1/\theta)$. Frailty accounts for unobserved factors that affect the baseline hazard across units, analogous to random effects in panel models.
\end{definition}

\subsection*{Integration with Panel Causal Designs}

Classic studies model takeoff hazards for consumer durables and international settings, documenting systematic roles for economics, culture, and innovativeness \citet{golder1997will,tellis2003international,hauser2007research}.

Panel causal designs complement hazards when takeoff events are staggered. Event-time analyses trace pre-trend diagnostics and dynamic post-takeoff effects on outcomes. Synthetic control or factor methods provide counterfactual paths for early adopters versus later markets.

Together, hazard models answer \emph{when} takeoff occurs, while panel designs answer \emph{what} changes after takeoff and by how much under credible identification.
