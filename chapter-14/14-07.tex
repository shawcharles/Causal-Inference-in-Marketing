\section{Dynamics with Continuous Intensity}
\label{sec:continuous-dynamics}

Marketing responses to continuous treatments (price, ad spend) are rarely instantaneous; they propagate over time. \citet{xiao2025causal} formalise this using the \textbf{Average Causal Response (ACR)}.

\begin{definition}[Average Causal Response]\label{def:acr}
For a continuous treatment history $\underline{d}_{it}$, the Average Causal Response at time $t$ to a change in treatment at time $t$ is:
\[
\text{ACR}_t(\underline{d}_{it}) = \frac{\partial \mathbb{E}[Y_{it}(\underline{d}_{it})]}{\partial d_{it}}.
\]
This measures the sensitivity of the average potential outcome to a unit perturbation in the current dose, holding past doses constant. It is the continuous analogue of the instantaneous treatment effect.
\end{definition}

\begin{theorem}[Identification via Generalised TWFE]\label{thm:acr-identification}
Under a \textit{Linear Equi-Confounding} assumption (which generalises parallel trends to continuous intensities), the ACR is identified by a Generalised Two-Way Fixed Effects estimator. Unlike the binary case, where TWFE weights can be negative, \citet{xiao2025causal} derive a weighting function that ensures the ACR is a convex combination of unit-level marginal effects, provided the treatment intensity variance is stable over time.
\end{theorem}

Practically, this justifies the use of distributed lag models with continuous regressors:
\[
Y_{it} = \sum_{s=0}^L \beta_s D_{i,t-s} + \alpha_i + \gamma_t + \varepsilon_{it},
\]
interpreting $\beta_s$ as the Average Causal Response at lag $s$ (the impulse response function).

\subsection*{Bridging Static and Dynamic Estimands}

This framework bridges the gap between static dose-response curves (Section~\ref{sec:continuous-estimands}) and the dynamic binary estimators of Chapter~\ref{ch:dynamics}. For marketing, it provides the rigorous justification for calculating ``Long-Run ROI'' from continuous ad-spend models.
