
\section{Motivation and Setup}
Many marketing interventions are not binary. A price change is a continuous treatment. An advertising campaign has a budget. A discount has a percentage. Similarly, outcomes are often not continuous. Customers make binary choices, they convert a certain number of times, or their purchasing is censored at zero. These features of the data require us to move beyond the simple models and develop a richer toolkit.

The potential outcomes framework from Chapter~\ref{ch:frameworks} extends naturally to this setting. For any given 'dose' of a treatment, $d$, we can define a potential outcome $Y_{it}(d)$. As always, identification rests on credible assumptions, which we must state clearly. The methods we develop here complement the design-based approaches in previous chapters and are especially useful when the intensity of a treatment is the key source of variation.

Throughout, we link estimands to identification, estimation, and inference, situating practice within modern panel frameworks \citet{arkhangelsky2024causal,arkhangelsky2024causal}. We emphasise transparent diagnostics for overlap and support in dose, stability across folds, and reconciliation with design-based benchmarks.
