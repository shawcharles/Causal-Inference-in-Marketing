\section{Marketing Applications}
\label{sec:continuous-applications}

We illustrate continuous treatment methods with brief vignettes. Each case states the estimand, identification route (GPS, DML, factor, or IV), tuning and overlap checks, clustered or bootstrap inference, and policy mapping.

\subsection*{Price-Response Curves}

Price-response curves yield dose-optimal pricing conditional on competition and seasonality. The estimand is the ADRF mapping price to demand, with identification via GPS conditioning on competitor prices, store characteristics, and time effects.

\subsection*{Advertising Intensity}

Advertising intensity influences brand search and sales with diminishing returns captured by smooth ADRFs and ad-stock dynamics. DML with flexible learners handles high-dimensional media mix covariates while respecting the temporal structure of carryover effects.

\subsection*{Impression Frequency}

Impression frequency and conversions motivate distributional effects when most users do not convert while a few do. Quantile dose-response functions or distributional treatment effects reveal heterogeneity masked by mean effects.

\subsection*{Loyalty Programme Intensity}

Loyalty points accrual rates affect purchase frequency with thresholds and saturation. The dose-response curve may exhibit non-monotonicities as customers game threshold rewards, requiring flexible basis functions to capture these patterns.
