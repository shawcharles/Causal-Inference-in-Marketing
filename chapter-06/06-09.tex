\section{Marketing Applications}
\label{sec:sc-marketing}

Synthetic control methods are particularly well-suited to marketing applications where a single unit or a small number of units receive treatment and transparency and executive communication are priorities. This section develops synthetic control designs in four common marketing settings: flagship city campaign, exclusive retail partnership, regional regulation, and platform entry, and then discusses an empirical application linking offline advertising to online chatter.

\subsection*{Flagship City Campaign}

Flagship city campaigns provide a canonical setting for synthetic control. A consumer packaged goods brand launches a major television advertising campaign in a designated market area (DMA), investing heavily in television spots over a six-month period. The goal is to estimate the causal effect on sales. The treated DMA is the flagship market. The initial donor pool consists of 40 other DMAs that did not receive the campaign. Pre-treatment data span three years (36 months), which gives a long window to establish the baseline trajectory and seasonal pattern. Post-treatment data span 12 months, which allows us to study both immediate and medium-run effects.

Design starts with donor curation. DMAs with incomplete data, idiosyncratic local shocks, or overlapping media interventions during the pre-treatment window are removed from the donor pool. Neighbouring DMAs that share substantial media spillover are flagged for later sensitivity analysis. The analyst then selects predictors: lagged monthly sales in the pre-treatment period, along with covariates such as population, median income, retail distribution, and competitive intensity. This mirrors the general guidance from Section~\ref{sec:sc-construction}. The synthetic control should match the treated unit's path over time using lagged outcomes, while the structural covariates help anchor long-run levels.

The synthetic control is constructed using these pre-treatment outcomes and covariates. The optimisation typically produces weights that assign substantial mass to three or four donor DMAs with similar demographics and pre-treatment sales trajectories. In the benchmark specification the pre-treatment RMSPE is around 5\% of the treated DMA's average sales (well below the 10\% threshold for excellent fit from Section~\ref{sec:sc-diagnostics}), and the gap plot shows that the synthetic control tracks the treated DMA closely across all 36 pre-treatment months rather than only in a short sub-window. If the fit were weak, or if the synthetic control matched only the last year of data but not the earlier part of the pre-period, the analyst would revisit donor curation and predictor choice, or conclude that the design does not support a credible synthetic control.

In-space placebos are then conducted for all donor DMAs using the placebo-based inference tools from Section~\ref{sec:sc-inference}. For each DMA the analyst re-assigns treatment at the campaign launch date and re-estimates a synthetic control using the same predictors and donor pool. In a successful design the treated DMA's post-treatment RMSPE ratio is noticeably larger than those of the placebo units. In our example it is the largest among all units, yielding a rank-based p-value of $p = 1/41 \approx 0.024$. The gap plot overlaying placebo gaps shows that the treated DMA's gap is an outlier, standing above the cloud of placebo gaps. If instead the treated DMA's RMSPE ratio lay in the middle of the distribution, the analyst would be cautious about interpreting the observed post-treatment gap as evidence of an effect.

This design rests on an identifying assumption that, conditional on matching the pre-treatment sales path and covariates, there are no DMA-specific shocks that line up exactly with the campaign launch for the treated DMA but not for the donors. Spillovers are an important potential violation if the campaign generates awareness in neighbouring DMAs through media overlap or customer mobility. To probe this, donor DMAs geographically adjacent to the treated DMA are excluded in a sensitivity analysis. If the estimated effect and RMSPE ratios remain stable after excluding adjacent donors, this supports the view that spillovers are not driving the results. Additional falsification checks examine whether other campaign-like events or distribution changes occur in the treated or donor DMAs around the launch date.

Once design and diagnostics are satisfactory, the analyst translates the cumulative effect, defined as the sum of monthly gaps over the 12 post-treatment months, into incremental revenue and compares it to campaign cost to estimate return on investment. The final report combines gap plots, weight tables, placebo distributions, and a narrative interpretation that explains any ramp-up dynamics and long-run persistence. This structure illustrates how synthetic control can support an expansion decision to other DMAs while making the identifying assumptions and diagnostic evidence explicit.

\subsection*{Exclusive Retail Partnership}

Exclusive retail partnerships illustrate synthetic control in settings with strategic, non-replicable interventions. Consider a retailer that signs an exclusive partnership with a popular brand, agreeing to carry the brand's full product line in exchange for favourable terms. The partnership is piloted in a single flagship store. The goal is to estimate the effect on store-level revenue and foot traffic. The treated store is the flagship. The donor pool consists of 50 stores in the same chain with similar format, size, and location characteristics. Pre-treatment data span two years (24 months). Post-treatment data span one year (12 months).

The synthetic control is constructed using pre-treatment monthly revenue, foot traffic, and covariates such as store size, local demographics, and proximity to competitors. Because the pilot store is chosen strategically, one concern is that management selects a store that is already on an improving trajectory for unobserved reasons. The analyst therefore examines pre-treatment trends carefully. In a credible design the synthetic control matches both the level and the slope of revenue and traffic for the flagship store, and pre-treatment RMSPE is small. If the treated store already exhibits a strong upward drift that cannot be reproduced by any convex combination of donors, the synthetic control design is unlikely to recover the causal effect of the partnership. More subtly, pre-treatment trend matching cannot rule out selection on anticipated future performance: if management chose the flagship store because they expected it to outperform, the SC estimate will be upward-biased even with perfect pre-treatment fit.

In the benchmark specification the post-treatment gap is positive and growing over time, indicating that the partnership increases revenue with a ramp-up period as customers discover the new product line. In-space placebos again show that the treated store's gap is in the top 5\% of placebo gaps, which, together with the pre-treatment fit, supports the presence of an effect. The analysis includes a sensitivity check excluding stores that may have been indirectly affected by the partnership, for example stores that experience customer substitution from the flagship. If the estimated effect is robust to these exclusions, the analyst gains confidence that the effect is not driven by within-chain reallocation rather than genuine incremental demand.

The retailer uses the synthetic control analysis to assess whether to expand the partnership to additional stores. The gap plot and the cumulative revenue gain are presented to executives, who value the transparency and the visual evidence. The blueprint from this example underscores that, in strategic pilot settings, synthetic control is most convincing when strong pre-treatment alignment is combined with robustness to alternative donor pools that address substitution and other selection concerns.

\subsection*{Regional Regulation}

Regional regulations provide a setting where synthetic control is the natural method because treatment is applied to a single geographic or administrative unit by external policy rather than firm choice. Suppose a city enacts a regulation restricting retail promotions, for example banning loss-leader pricing or limiting discount frequency. The goal is to estimate the effect on retail prices and sales. The treated city is the regulated market. The donor pool consists of 30 unregulated cities in the same country. Pre-treatment data span three years (36 months). Post-treatment data span two years (24 months).

The synthetic control is constructed using pre-treatment monthly prices and sales, along with city-level covariates such as population, income, and retail concentration. The pre-treatment fit is very tight, and the gap plot for prices and sales shows no systematic divergence before the regulation comes into force. After the regulation, the synthetic control counterfactual tracks what would have happened in the absence of the policy, and the post-treatment gap shows that prices increase and sales decline, consistent with economic theory. In-space placebos provide strong evidence that the observed pattern is unusual relative to the donor cities.

This design relies on the assumption that, once we control for pre-treatment trends and covariates, there are no other city-specific shocks coinciding with the regulation that move prices and sales in the treated city but not in the donors. Sensitivity analyses therefore check whether the effect could be driven by concurrent macroeconomic or sectoral shocks. The analyst examines whether donor cities experienced similar changes in national unemployment, inflation, and sector-wide shocks, and verifies that these are either common across treated and donor cities or absorbed by the synthetic control. The absence of parallel breaks in the donor cities supports a causal interpretation of the regulation's effect.

The analysis is used by policymakers to assess the welfare implications of the regulation and by industry stakeholders to argue for or against expanding the regulation to other markets. The transparency of the synthetic control method, which makes clear exactly which cities contribute to the counterfactual and how closely they match the treated city, helps the analysis speak to both audiences.

\subsection*{Platform Entry with Spillovers}

This example is deliberately constructed as a cautionary case: it illustrates the challenges of synthetic control when interference is present and shows when the method reaches its limits. A food delivery platform enters a major metropolitan market, partnering with restaurants and launching a marketing campaign to attract users. The goal is to estimate the direct effect on restaurant revenues in the treated city and the spillover effect on nearby restaurants and markets. The treated market is the city where the platform entered. Potential donors are 25 cities without platform entry. However, spillovers are likely. Restaurants within the treated city that do not join the platform may experience demand shifts, and restaurants in adjacent cities may be affected by cross-border customer mobility.

In a first pass the analyst applies the standard synthetic control recipe. The synthetic control is constructed using pre-treatment monthly restaurant revenues and city-level covariates, and donor cities within a buffer distance of the treated city are excluded to reduce contamination. The pre-treatment fit is acceptable and the post-treatment gap suggests a positive effect on restaurant revenues in the treated city. However, in-space placebos reveal that some donor cities exhibit unusual post-treatment gaps that align with the timing of entry, even though they were nominally untreated. This pattern is a warning sign that spillovers may have contaminated the donor pool despite buffering.

Rather than treating this as a nuisance to be fixed by ever more aggressive donor pruning, the analyst reframes the problem using the explicit spillover models developed in Chapter~\ref{ch:spillovers}. The refined analysis redefines the donor pool to include only cities far from the treated city, models exposure as a function of distance and platform penetration, and estimates direct and spillover effects jointly. In this framework the synthetic control for the treated city provides one ingredient for the counterfactual path, but identification of spillovers relies on the richer exposure mapping and additional structure from the interference setting.

In a representative application the analysis concludes that platform entry raises revenues for participating restaurants in the entry city, generates positive spillovers for some non-participating restaurants through category expansion, and may reduce revenues in adjacent markets through customer substitution. The example is deliberately constructed to show that vanilla synthetic control, even with buffering and placebos, can struggle in the presence of substantial interference, and that credible answers require methods that treat spillovers as a first-order design feature rather than a minor complication.

These four applications illustrate the versatility and transparency of synthetic control methods in marketing and provide blueprints for designing and diagnosing studies in practice. The method provides credible counterfactuals, intuitive visualisations, and design-based inference that communicates evidence clearly to technical and non-technical stakeholders. The key is to tailor donor curation, predictor selection, and the diagnostic workflow to the substantive context, and to document assumptions and sensitivity analyses transparently.

\subsection*{Offline Advertising and Online Chatter}

Tirunillai and Tellis apply synthetic control to measure how offline television advertising affects online user-generated content \citep{tirunillai2017offline}. The setting is a major consumer brand launching a television advertising campaign. The outcome is online chatter on social media and review platforms, measured along multiple dimensions including popularity (volume of mentions), negativity (sentiment), and visibility (reach). The challenge is that campaign timing is endogenous to brand strategy. Brands launch campaigns when they anticipate favourable conditions or need to counteract negative trends.

The authors construct synthetic controls for treated brands by matching on pre-campaign chatter metrics over 24 weeks. Donor brands are selected from the same product category but did not launch campaigns during the study window. Weights are chosen to minimise pre-treatment root mean squared prediction error across all chatter dimensions simultaneously. The method reveals that television advertising increases online chatter popularity by 15\% and reduces negativity by 8\%, with effects persisting for three weeks before decaying to baseline.

This application illustrates how synthetic control can mitigate concerns about endogenous campaign timing when we observe a long, well-fitted pre-treatment path and have a rich donor pool from the same product category. The design is most credible when there are no emerging pre-trends in chatter immediately before the campaign and no brand-specific shocks coinciding with launch that donor brands cannot reproduce. The multi-dimensional outcome structure requires careful aggregation. The authors report separate effects for each chatter dimension and conduct in-space placebos by applying the method to each donor brand as if it were treated. The treated brand's post-treatment RMSPE ranks in the top decile, supporting the inference that the observed effects are unusual relative to the donor pool. The analysis links offline advertising investment to online engagement metrics, demonstrating cross-channel effects that inform integrated marketing strategies.
