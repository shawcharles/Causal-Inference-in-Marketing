\section{Workflow Checklist}
\label{sec:sc-workflow}

This section provides a compact end-to-end protocol for conducting synthetic control analyses in marketing panels. The workflow integrates design, construction, diagnostics, inference, and reporting, ensuring that conclusions are credible and transparent. Following these nine steps systematically helps practitioners avoid common pitfalls and produce analyses that withstand scrutiny.

\textbf{Step 1: Define the treated unit and intervention time.} Identify the treated unit (or units) and the calendar period in which treatment begins. If treatment is announced in advance, decide whether the intervention time is the announcement date or the implementation date. Base this decision on whether anticipation is plausible. Document the treatment and its timing clearly. Provide institutional context on why the unit received treatment and how it was implemented.

\textbf{Step 2: Curate the donor pool.} Assemble the donor pool by excluding units that received treatment themselves, units that were contaminated by spillovers, and units that are fundamentally incomparable to the treated unit. Document donor selection criteria (for example, geographic region, market size, operational characteristics). Provide summary statistics comparing donors to the treated unit. If spillovers are plausible, define a buffer zone and exclude donors within the buffer. Report the final donor pool size and the characteristics of included donors.

\textbf{Step 3: Select predictors and pre-treatment window.} Define the pre-treatment window (all periods before the intervention time). Select predictors including pre-treatment outcomes (monthly or quarterly outcomes for the full pre-treatment window) and key covariates (pre-treatment or time-invariant characteristics). Avoid including post-treatment covariates or covariates affected by treatment. Document the predictor set and the rationale for including each predictor. If the pre-treatment period is short, consider using longer lags of outcomes or differenced outcomes to capture trends.

\textbf{Step 4: Fit synthetic control and assess pre-treatment fit.} Solve the weight optimisation problem to obtain synthetic control weights $w_j$ for $j \in \mathcal{J}$. Compute the pre-treatment RMSPE and assess whether the fit is tight relative to the scale of the outcome and the variability across units. Plot the treated unit's pre-treatment outcomes and the synthetic control's pre-treatment outcomes on the same axes to visualise the fit. Check predictor balance by comparing the predictor values for the treated unit to the weighted average of predictor values for the donors. If the fit is poor or predictor balance is weak, revisit the donor pool or the predictor set. Iterate until satisfactory.

Report the synthetic control weights explicitly. Identify which donors receive substantial weight and assess whether the weights are concentrated on a few units or distributed broadly. Sparse weights (few donors with large weights) indicate that the treated unit is well-matched by a small set of donors. Diffuse weights may indicate that no single donor is a close match, and the synthetic control relies on averaging across many units.

\textbf{Step 5: Produce gap plots and estimate post-treatment effects.} Extend the plot of outcomes into the post-treatment period, showing the treated unit's observed outcomes and the synthetic control's counterfactual outcomes. Compute the gap (treated minus synthetic) for each post-treatment period and plot the gap over time. Compute the cumulative gap (sum of gaps over the post-treatment window) and the average gap (mean over the post-treatment window) as summary measures of the treatment effect. Interpret the gap plot carefully. Does the gap open immediately after treatment, or is there a ramp-up period? Does the gap persist, or does it decay?

\textbf{Step 6: Run in-space and in-time placebos (Section~\ref{sec:sc-inference}).} Conduct in-space placebos by applying the synthetic control method to each donor unit as if it received treatment at the intervention time. Compute post-treatment gaps and RMSPE ratios for all placebo units. Plot the treated unit's gap alongside placebo gaps to assess whether the treated unit's gap is an outlier. Compute the rank-based p-value and report it.

Conduct in-time placebos by applying the synthetic control method using a pseudo-intervention date in the middle of the pre-treatment period. Assess whether the pseudo post-treatment gap is near zero. Report placebo results transparently, including plots and rank statistics.

\textbf{Step 7: Select inference procedure and report uncertainty.} Choose an inference procedure based on the sample size and the design. If the donor pool is large ($N \geq 20$), rank-based p-values provide adequate resolution (with $N = 20$, the smallest achievable p-value is approximately 0.05). If the donor pool is small, focus on the magnitude and persistence of the gap rather than on binary significance tests. Report RMSPE ratios, ranks, and p-values. Interpret them in the context of the substantive magnitude of the effect. If multiple outcomes or multiple post-treatment periods are tested, discuss multiplicity and consider adjustments or joint tests.

\textbf{Step 8: Conduct sensitivity analyses (Section~\ref{sec:sc-diagnostics}).} Vary the donor pool by excluding influential donors one at a time (leave-one-donor-out) and re-estimating the synthetic control. Report the range of post-treatment gaps across leave-one-donor-out specifications. Vary the predictor set by including or excluding specific covariates or pre-treatment periods, and re-estimate the synthetic control. If spillovers or anticipation are concerns, vary the buffer zone or the intervention date and assess robustness. Construct a specification curve showing the distribution of estimated effects across defensible modelling choices. If estimates cluster tightly, conclusions are robust. If estimates vary widely, document the sensitivity and discuss which specifications are most credible.

\textbf{Step 9: Document assumptions and report results.} Prepare a comprehensive report that includes the research question, the treated unit and intervention time, the donor pool curation process, the predictor set, the pre-treatment fit metrics, the synthetic control weights, the gap plot, the placebo results, the inference procedure, the sensitivity analyses, and the substantive interpretation. Articulate the identification assumptions clearly. State the no anticipation, no interference, and stable predictor-outcome relationship assumptions explicitly. Provide evidence that they are plausible. Discuss threats to validity (poor fit, contamination, structural breaks) and the robustness of conclusions. If the analysis was pre-registered, document any deviations from the pre-analysis plan and provide justification.

Translate the estimated gap into business metrics relevant for marketing decisions. Express effects in terms of incremental revenue, market share changes, or return on investment. Discuss the ramp-up dynamics, persistence, and long-run implications. Relate findings to decision-relevant questions such as whether to expand the intervention, adjust its design, or discontinue it.

Provide replication materials including cleaned data (or simulated data if proprietary), analysis scripts, and documentation of software versions. This enables readers to verify results and to conduct alternative analyses.

By following this workflow, practitioners can conduct synthetic control analyses that are transparent, rigorous, and aligned with modern best practices. The workflow integrates design-based reasoning, careful donor curation, rigorous diagnostics, permutation-based inference, and \index{sensitivity analysis}sensitivity analysis. This ensures that conclusions are credible and that assumptions are articulated and assessed. The result is causal evidence that withstands scrutiny and informs strategic decisions with confidence.
\index{synthetic control|)}
