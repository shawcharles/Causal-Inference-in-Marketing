\section{Bias Decomposition Framework}
\label{sec:bias-decomposition}

To rigorously diagnose threats, we first decompose the potential bias of an estimator. This decomposition links specific assumption violations to distinct components of error, providing a roadmap for sensitivity analysis.

\begin{definition}[Bias Decomposition]\label{def:bias-decomposition}
Let $\hat{\tau}$ be an estimator of the causal effect $\tau_0$. The bias decomposes as:
\[
\text{Bias}(\hat{\tau}) = \mathbb{E}[\hat{\tau}] - \tau_0 = \underbrace{B_{\text{confound}}}_{\text{confounding}} + \underbrace{B_{\text{PT}}}_{\text{parallel trends}} + \underbrace{B_{\text{meas}}}_{\text{measurement}} + \underbrace{B_{\text{spill}}}_{\text{spillovers}},
\]
where each component corresponds to a specific assumption violation:
\begin{itemize}
\item $B_{\text{confound}}$ captures bias from unobserved confounding, violating unconfoundedness.
\item $B_{\text{PT}}$ reflects bias from differential trends, violating parallel trends.
\item $B_{\text{meas}}$ arises from measurement error in treatment or outcome.
\item $B_{\text{spill}}$ accounts for bias from interference or spillovers, violating SUTVA.
\end{itemize}
\end{definition}

Ideally, research design eliminates these terms. When elimination is impossible, we aim to bound them.

\subsection*{The Estimand-Estimator Gap}

A fundamental source of bias is the mismatch between the theoretical target and the empirically identified quantity.

\begin{definition}[Estimand-Estimator Gap]\label{def:estimand-gap}
Let $\tau^*$ denote the target estimand (e.g., ATT, ATE) and $\tilde{\tau}$ denote the population quantity identified by the estimation strategy under the maintained assumptions. The estimand-estimator gap is:
\[
\Delta = \tau^* - \tilde{\tau}.
\]
This gap is zero under correct specification and valid assumptions. Threats to validity create $\Delta \neq 0$ by causing the identified quantity to differ from the causal effect of interest.
\end{definition}
