\chapter*{Preface}
\addcontentsline{toc}{chapter}{Preface}

Marketing measurement stands at a crossroads. Each year, firms allocate billions to advertising, pricing experiments, and customer acquisition---yet the methods used to evaluate these investments remain fragmented and, too often, causally incoherent.

On one side sits the tradition of Marketing Mix Modelling: time-series regressions, ad-stock transformations, and attribution heuristics refined over decades of industry practice. On the other lies the econometric revolution in causal inference---Synthetic Control, Difference-in-Differences, factor models---developed largely in policy evaluation contexts that assume data structures rarely encountered in commercial settings.

This fragmentation carries real costs. Traditional MMM struggles to distinguish correlation from causation when campaigns overlap, markets interfere, and consumer behaviour shifts. Academic estimators, designed for clean quasi-experiments, break down when confronted with the sparsity, noise, and high dimensionality of modern marketing data. Practitioners are left to choose between tools that are practical but causally suspect, or rigorous but operationally brittle.

This book refuses that trade-off. It develops a unified framework---grounded in panel econometrics and the potential outcomes tradition---that meets marketing data on its own terms. The methods presented here, from Synthetic Difference-in-Differences to Double Machine Learning, represent a convergence: the identification discipline of causal inference joined to the flexibility demanded by commercial reality.

The exposition is deliberately practitioner-oriented without sacrificing rigour. Every estimator is derived from first principles, every assumption stated explicitly, every diagnostic motivated by the ways real campaigns violate textbook conditions. Code accompanies theory; applications accompany proofs.

This monograph is written for the data scientist who knows that correlation is not causation but needs the tools to prove it---and for the econometrician ready to apply their training to the high-stakes questions of budget allocation, incrementality, and return on investment.

\bigskip

\noindent
Charles Shaw\\
London, 2025


