\section{Computational Methods for Large-Scale Problems}
\label{sec:computation}

\subsection*{Scalability Challenges}

Large-scale marketing panels pose computational challenges. A panel with 10,000 products, 1,000 stores, and 100 weeks contains one billion cells. Standard matrix completion algorithms (nuclear norm minimisation, SVD) have computational complexity $O(N^2 T)$ or higher, making them infeasible for such large panels.

Memory constraints also bind. Storing a dense matrix of size $10,000 \times 100,000$ (products by store-weeks) requires 8 gigabytes (assuming 8 bytes per entry). Most marketing panels are sparse, but standard algorithms do not exploit sparsity. Sparse storage (e.g., CSR format) reduces memory to $O(|\Omega|)$, where $|\Omega|$ is the number of observed entries.

\subsection*{Efficient Algorithms}

\begin{proposition}[Computational Complexity]\label{prop:complexity}
The computational complexity of key algorithms is:
\begin{enumerate}[(i)]
    \item \textbf{Nuclear norm minimisation via SVD:} $O(\min(N^2 T, N T^2))$ per iteration;
    \item \textbf{Soft-Impute:} $O(|\Omega| R + (N + T) R^2)$ per iteration;
    \item \textbf{Alternating Least Squares:} $O(|\Omega| R^2 + (N + T) R^3)$ per iteration;
    \item \textbf{Stochastic Gradient Descent:} $O(B R)$ per iteration, where $B$ is the mini-batch size;
    \item \textbf{MCMC (Gibbs sampling):} $O((N + T) R^2)$ per sample.
\end{enumerate}
\end{proposition}

\begin{tcolorbox}[colback=green!5!white,colframe=green!50!black,title=Algorithm: Soft-Impute for Matrix Completion]

\paragraph{Input:} Observed entries $\{Y_{it}\}_{(i,t) \in \Omega}$, regularisation $\lambda$, rank $R$, tolerance $\epsilon$.

\paragraph{Initialise:} Set $\hat{\mathbf{L}}^{(0)} = \mathbf{0}_{N \times T}$.

\paragraph{For $k = 1, 2, \ldots$ until convergence:}
\begin{enumerate}
\item \textbf{Fill in missing entries:} Create completed matrix
\[
\tilde{\mathbf{Y}}^{(k)} = P_\Omega(\mathbf{Y}) + P_{\Omega^c}(\hat{\mathbf{L}}^{(k-1)}),
\]
where $P_\Omega$ projects onto observed entries and $P_{\Omega^c}$ onto missing entries.

\item \textbf{Soft-thresholded SVD:} Compute SVD $\tilde{\mathbf{Y}}^{(k)} = \mathbf{U} \boldsymbol{\Sigma} \mathbf{V}^\top$ and apply soft-thresholding:
\[
\hat{\mathbf{L}}^{(k)} = \mathbf{U} \, \text{diag}((\sigma_r - \lambda)_+) \, \mathbf{V}^\top,
\]
where $(x)_+ = \max(0, x)$.

\item \textbf{Check convergence:} If $\|\hat{\mathbf{L}}^{(k)} - \hat{\mathbf{L}}^{(k-1)}\|_F / \|\hat{\mathbf{L}}^{(k-1)}\|_F < \epsilon$, stop.
\end{enumerate}

\paragraph{Output:} Low-rank estimate $\hat{\mathbf{L}}$.
\end{tcolorbox}

\paragraph{Why Soft-Impute is efficient.} Soft-Impute exploits sparsity by operating only on observed entries. The SVD step uses truncated (partial) SVD, computing only the top $R$ singular values. Complexity is $O(|\Omega| R + (N + T) R^2)$ per iteration—for sparse panels with $|\Omega| \ll NT$, this is much faster than dense SVD.

\paragraph{Alternating Least Squares (ALS).} ALS alternates between updating loadings (fixing factors) and updating factors (fixing loadings). Each step is a least squares problem:
\[
\hat{\boldsymbol{\lambda}}_i = \left( \sum_{t: (i,t) \in \Omega} f_t f_t^\top + \lambda \mathbf{I} \right)^{-1} \sum_{t: (i,t) \in \Omega} Y_{it} f_t.
\]
ALS is particularly effective for tensor completion (Section~\ref{sec:tensor-completion}).

\begin{theorem}[Convergence of Alternating Least Squares]\label{thm:als-convergence}
For CP tensor decomposition via ALS, under mild regularity conditions:
\begin{enumerate}[(i)]
    \item The sequence of objective values $\{Q^{(k)}\}$ is non-increasing and converges;
    \item Every limit point is a stationary point of the objective function;
    \item With probability $1 - \delta$, convergence to an $\epsilon$-stationary point requires $K = O(\kappa^2 \log(1/\epsilon))$ iterations, where $\kappa$ is the condition number.
\end{enumerate}
\end{theorem}

\paragraph{Stochastic Gradient Descent (SGD).} SGD updates using mini-batches of observed entries:
\[
\lambda_i^{(k+1)} = \lambda_i^{(k)} - \eta_k \sum_{t \in B_k} (Y_{it} - \lambda_i^{(k)\top} f_t^{(k)}) f_t^{(k)},
\]
where $B_k$ is a mini-batch of size $B$. SGD handles datasets that do not fit in memory but converges more slowly than batch methods.

\subsection*{Algorithm Selection Guide}

Table~\ref{tab:algorithm-comparison} compares algorithms across key dimensions.

\begin{table}[htbp]
\begin{tighttable}
\centering
\caption{Algorithm Comparison for Matrix Completion}
\label{tab:algorithm-comparison}
\begin{tabularx}{\textwidth}{Y Y Y Y Y}
\toprule
\textbf{Algorithm} & \textbf{Complexity} & \textbf{Memory} & \textbf{Sparsity} & \textbf{When to Use} \\
\midrule
Nuclear norm & $O(N^2 T)$ & $O(NT)$ & No & Small dense panels \\
\addlinespace
Soft-Impute & $O(|\Omega| R)$ & $O(|\Omega|)$ & Yes & Sparse panels \\
\addlinespace
ALS & $O(|\Omega| R^2)$ & $O(|\Omega|)$ & Yes & Tensors; known rank \\
\addlinespace
SGD & $O(BR)$ & $O(B)$ & Yes & Very large panels \\
\addlinespace
MCMC & $O((N+T)R^2)$ & $O(NT)$ & No & Uncertainty needed \\
\bottomrule
\end{tabularx}
\end{tighttable}
\end{table}

\begin{tcolorbox}[colback=green!5!white,colframe=green!50!black,title=Practical Guidance: Algorithm Selection]

\paragraph{Decision tree for algorithm selection:}
\begin{enumerate}
\item \textbf{Need uncertainty quantification?} $\rightarrow$ Use MCMC (Section~\ref{sec:bayesian-matrix}).
\item \textbf{Panel fits in memory?}
  \begin{itemize}
  \item If sparse ($>$90\% missing) $\rightarrow$ Use Soft-Impute.
  \item If dense with known rank $\rightarrow$ Use ALS.
  \item If dense, rank unknown $\rightarrow$ Use nuclear norm with cross-validation.
  \end{itemize}
\item \textbf{Panel does not fit in memory?} $\rightarrow$ Use SGD with mini-batches.
\item \textbf{Tensor structure?} $\rightarrow$ Use CP-ALS (Section~\ref{sec:tensor-completion}).
\end{enumerate}
\end{tcolorbox}

\subsection*{Approximation Methods}

Sketching and randomised algorithms provide approximate solutions with provable error bounds.

\paragraph{Randomised SVD.} Instead of computing the full SVD, randomised SVD projects the matrix onto a random low-dimensional subspace:
\begin{enumerate}
\item Generate random matrix $\boldsymbol{\Omega} \in \mathbb{R}^{T \times (R + p)}$, where $p$ is oversampling.
\item Compute $\mathbf{Q} = \text{orth}(\mathbf{Y} \boldsymbol{\Omega})$ (orthonormal basis for range).
\item Compute SVD of $\mathbf{Q}^\top \mathbf{Y}$ and project back.
\end{enumerate}
Complexity: $O(NTR)$ vs $O(NT^2)$ for exact SVD. Approximation error is bounded by $\|\mathbf{Y} - \hat{\mathbf{Y}}\|_F \leq (1 + \epsilon) \|\mathbf{Y} - \mathbf{Y}_R\|_F$, where $\mathbf{Y}_R$ is the best rank-$R$ approximation.

\paragraph{Low-rank approximations.} Trade accuracy for speed by solving with smaller rank than the true rank. Error is bounded by the difference between true and approximate rank. Useful when true rank is unknown.

\subsection*{Software and Implementation}

\paragraph{R.}
\begin{itemize}
\item \texttt{softImpute}: Soft-impute algorithm. Handles matrices with millions of entries.
\item \texttt{MatrixCompletion}: Nuclear norm minimisation.
\item \texttt{tensorBF}: Bayesian tensor factorisation.
\end{itemize}

\paragraph{Python.}
\begin{itemize}
\item \texttt{fancyimpute}: Multiple imputation algorithms including Soft-Impute.
\item \texttt{scikit-learn}: \texttt{NMF} for matrix factorisation.
\item \texttt{tensorly}: Tensor decomposition (CP, Tucker).
\item \texttt{implicit}: ALS for large sparse matrices (recommender systems).
\end{itemize}

\paragraph{Julia.}
\begin{itemize}
\item \texttt{LowRankModels.jl}: Generalised low-rank models.
\item \texttt{TensorToolbox.jl}: Tensor decomposition.
\end{itemize}

\subsection*{Benchmarking}

Benchmarking studies compare speed and accuracy of different algorithms. The following results are illustrative; actual performance depends on sparsity, rank, and hardware.

\begin{table}[htbp]
\begin{tighttable}
\centering
\caption{Illustrative Benchmark Results (Intel i7, 32GB RAM)}
\label{tab:benchmarks}
\begin{tabularx}{\textwidth}{Y Y Y Y}
\toprule
\textbf{Algorithm} & \textbf{Sparsity} & \textbf{Time (relative)} & \textbf{MSE (relative)} \\
\midrule
Nuclear norm & 1\% & 1.0x (baseline) & 1.0x \\
\addlinespace
Soft-Impute & 1\% & 0.1x (10x faster) & 1.05x \\
\addlinespace
ALS & 1\% & 0.15x & 1.02x \\
\addlinespace
Soft-Impute & 50\% & 0.5x & 1.0x \\
\addlinespace
ALS & 50\% & 0.2x & 1.0x \\
\bottomrule
\end{tabularx}
\end{tighttable}
\end{table}

\textbf{Caveats:} Performance depends heavily on matrix dimensions, sparsity pattern, rank, and implementation. For dense matrices, ALS is typically faster than Soft-Impute. For very sparse matrices ($>$99\% missing), Soft-Impute excels. For tensors, CP-ALS is typically 10--20x faster than Tucker decomposition. Always benchmark on your specific problem.
