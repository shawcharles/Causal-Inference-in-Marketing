\section{Randomisation Inference}
\label{sec:randomisation}
\index{randomisation inference|(}
\index{Fisher randomisation test}

For design-based causal inference, particularly with Synthetic Control (Chapter~\ref{ch:sc}) or small-N Difference-in-Differences, randomisation inference (RI) offers exact p-values without relying on large-sample asymptotics. The logic is simple: if the treatment had been assigned to a different unit, would we see the same effect?

\begin{definition}[Fisher Randomisation Test]\label{def:randomisation-inference}
Let $\mathcal{W}$ be the set of all feasible treatment assignments under the experimental design. The randomisation p-value for test statistic $T(\mathbf{Y}, \mathbf{W})$ is:
\[
p_{\text{rand}} = \frac{1}{|\mathcal{W}|} \sum_{\mathbf{w} \in \mathcal{W}} \mathbf{1}\{T(\mathbf{Y}, \mathbf{w}) \geq T(\mathbf{Y}, \mathbf{W}^{\text{obs}})\},
\]
where $\mathbf{W}^{\text{obs}}$ is the observed assignment. Under the sharp null $H_0: Y_i(1) = Y_i(0)$ for all $i$, the potential outcomes are fixed and only assignment varies.
\end{definition}

This method provides a gold standard for validity.

\begin{assumption}[Known assignment mechanism for randomisation inference]
\label{assump:inference-assignment}
The set of feasible treatment assignments $\mathcal{W}$ is known and correctly specified. Permutations respect the design (geo-stratification, switchback periods, partial interference structure). Assignment is unconfounded within the permutation class.
\end{assumption}

\begin{theorem}[Exact Size Control]\label{thm:randomisation-exact}
Under Assumption~\ref{assump:inference-assignment} (known assignment mechanism) and the sharp null hypothesis:
\begin{itemize}
    \item The randomisation p-value is exactly uniformly distributed: $P(p_{\text{rand}} \leq \alpha) = \alpha$ for all $\alpha \in [0,1]$.
    \item No distributional assumptions on outcomes are required.
    \item The test controls size exactly in finite samples for any $G$ or $N$.
\end{itemize}
When $|\mathcal{W}|$ is large, Monte Carlo approximation with $B$ random draws yields $\hat{p}_{\text{rand}}$ with simulation error $O(B^{-1/2})$.
\end{theorem}

The power of the test depends on the choice of statistic.

\begin{definition}[Choice of Test Statistic]\label{def:test-statistic-choice}
For randomisation inference, Table~\ref{tab:ri-statistics} summarises common test statistics and their properties.
\end{definition}

\begin{table}[htbp]
\begin{tighttable}
\centering
\caption{Test statistics for randomisation inference}
\label{tab:ri-statistics}
\begin{tabularx}{\textwidth}{Y Y Y}
\toprule
\textbf{Statistic} & \textbf{Formula} & \textbf{Properties} \\
\midrule
Difference in means & $T_{\text{diff}} = \bar{Y}_{\text{treat}} - \bar{Y}_{\text{control}}$ & Unstudentised; simple but less powerful \\
t-statistic & $T_t = (\bar{Y}_{\text{treat}} - \bar{Y}_{\text{control}}) / \hat{\text{se}}$ & Studentised; better power under heterogeneity \\
Rank statistic & $T_{\text{rank}} = \sum_{i: W_i = 1} R_i$ & Robust to outliers; distribution-free \\
\bottomrule
\end{tabularx}
\end{tighttable}
\end{table}

\begin{remark}[Randomisation Inference in Marketing Applications]\label{rem:ri-marketing}
Randomisation inference is the preferred method for:
\begin{enumerate}
\item \textbf{Synthetic control} (Chapter~\ref{ch:sc}): Permute treatment across donor units to generate placebo distribution.
\item \textbf{Geo-experiments with few markets} (Section~\ref{sec:geo-experiments}): When $G < 10$, RI provides exact inference where bootstrap fails.
\item \textbf{Switchback experiments} (Section~\ref{sec:platform-experiments}): Permute treatment periods within the design structure.
\item \textbf{Single-unit interventions}: When only one unit is treated, RI is the only valid approach.
\end{enumerate}
The key requirement is that the permutation class $\mathcal{W}$ correctly reflects the experimental design.
\end{remark}
\index{randomisation inference|}