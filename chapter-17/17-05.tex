\section{Support, Exposure, and Contamination}
\label{sec:support-exposure}

We must verify adequate data at each point in the analysis. For event studies (Chapter~\ref{ch:event}), this means tabulating the number of units and cohorts that contribute to each event-time bin. If only a few observations contribute to a particular bin, the estimate for that period will be noisy and potentially biased. The solution is often to bin sparse periods together, trading temporal precision for statistical power.

Exposure mapping (Definition~\ref{def:exposure-mapping} in Chapter~\ref{ch:spillovers}) defines treatment intensity for unit $i$ as a function of own treatment $D_{it}$ and neighbours' treatments $\{D_{jt} : j \in \mathcal{N}_i\}$. Common specifications include distance-based radii (exposure decays with geographic distance), network adjacency (exposure proportional to treated neighbours), and market-overlap matrices (exposure proportional to competitive overlap).

Buffers physically separate treated and control units by excluding units within a threshold distance (Chapter~\ref{ch:spillovers}). Re-estimating with varying buffer radii assesses whether spillovers bias the main estimate. When spillovers are expected but unmeasured, diagnostic tests for partial interference and design adjustments are necessary.

Contamination arises when control units receive indirect treatment through competitive responses, supply-chain linkages, or word-of-mouth. Diagnostic checks include testing for outcome changes in controls coincident with treatment rollout, examining pre-post trends in donor pools for synthetic control and SDID (Chapters~\ref{ch:sc}--\ref{ch:generalized-sc}), and inspecting geographic or network proximity. When contamination is detected, donor redesign, buffer inclusion, or switching to a spillover-aware estimand (Chapter~\ref{ch:spillovers}) mitigates bias.
