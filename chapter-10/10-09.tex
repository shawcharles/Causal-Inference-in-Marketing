\section{Workflow Checklist}
\label{sec:dynamics-workflow}

This section provides a compact, reproducible protocol for conducting dynamic treatment effect analyses in marketing panels. The workflow integrates design, estimation, diagnostics, inference, and reporting.

For estimation methods, see Section~\ref{sec:dynamics-estimation}. For identification, see Section~\ref{sec:dynamics-identification}. For diagnostics, see Section~\ref{sec:dynamics-diagnostics}. For inference, see Section~\ref{sec:dynamics-inference}.

\begin{tcolorbox}[colback=blue!5!white,colframe=blue!75!black,title=Box 10.2: Dynamic Effects Workflow Summary]
\small
\textbf{1. Define estimand:} Event-time effects, impulse responses, LRM, or half-life. Pre-specify window based on domain knowledge.

\textbf{2. Map support:} Plot treated and control units for each lag; bin sparse tails.

\textbf{3. Select estimator:} Event study (discrete), distributed-lag (continuous), or dynamic DiD (staggered).

\textbf{4. Run pre-trend checks:} Estimate leads; interpret significant leads as anticipation or design violations.

\textbf{5. Report dynamic profile:} Event-time plot with uniform confidence bands; LRM and half-life with bootstrap or delta-method SE.

\textbf{6. Inference:} Cluster-robust SE; joint tests; multiple testing adjustments.

\textbf{7. Diagnostics:} Follow protocol in Section~\ref{sec:dynamics-diagnostics}; placebo tests; sensitivity checks.
\end{tcolorbox}

\subsection*{Step 1: Define the Dynamic Estimand}

Clarify the substantive question and define the target estimand:
\begin{itemize}
\item \textbf{Event-time effects $\theta_k$:} Average effect at each lag relative to treatment adoption.
\item \textbf{Impulse responses $\beta_s$:} Marginal effect of treatment intensity at each lag.
\item \textbf{Long-run multiplier:} Cumulative effect over all lags.
\item \textbf{Half-life:} Time required for effects to decay to half.
\end{itemize}

Pre-specify the event-time window based on substantive knowledge about expected effect duration. Document the estimand, window, and rationale.

\subsection*{Step 2: Map Event-Time Support}

For each lag $k$, compute the number of treated and control observations:
\begin{enumerate}
\item Plot support per event time.
\item Flag lags where support is sparse ($< 10$ observations).
\item Decide whether to bin distant lags or restrict the window.
\item Document binning choices and rationale.
\item Conduct sensitivity analyses varying bin thresholds.
\end{enumerate}

\subsection*{Step 3: Select the Estimator}

Choose based on data structure and identification assumptions:
\begin{itemize}
\item \textbf{Event studies:} Discrete treatment; common or staggered adoption.
\item \textbf{Distributed-lag models:} Continuous treatment intensity; carryover expected.
\item \textbf{Dynamic DiD:} Staggered adoption; heterogeneity-robust estimates required.
\item \textbf{Continuous dose-response:} Treatment intensity varies; high-dimensional confounders.
\end{itemize}

Justify choice based on data structure and estimand. See Section~\ref{sec:dynamics-estimation}.

\subsection*{Step 4: Specify Leads, Lags, and Bins}

\begin{itemize}
\item \textbf{Lead window:} Lags $k < 0$ for pre-trend tests. Include at least 3--5 leads.
\item \textbf{Lag window:} Lags $k \geq 0$ for post-treatment effects.
\item \textbf{Reference category:} Typically $k = -1$ for event studies.
\item \textbf{Maximum lag $\bar{L}$:} For distributed-lag models, based on substantive knowledge or cross-validation.
\end{itemize}

Document the window, reference category, and rationale.

\subsection*{Step 5: Run Anticipation and Pre-Trend Checks}

\begin{enumerate}
\item Estimate the specification including leads ($k < 0$).
\item Plot lead coefficients with confidence intervals.
\item Conduct joint F-test that all lead coefficients are zero.
\item If F-test does not reject, parallel trends is supported.
\item If F-test rejects, diagnose: differential trends, anticipation, or confounders.
\item Consider covariate adjustment, factor models, or alternative controls.
\end{enumerate}

Report pre-trend test result and any adjustments. See Section~\ref{sec:dynamics-anticipation}.

\subsection*{Step 6: Choose Inference Procedure}

\begin{itemize}
\item \textbf{Cluster-robust SE:} By unit when serial dependence is strong.
\item \textbf{Two-way clustering:} By unit and time when common shocks are plausible.
\item \textbf{HAC (Newey-West):} When serial dependence decays with lag.
\item \textbf{Wild cluster bootstrap:} When $N < 30$ clusters.
\end{itemize}

Document choice and rationale. Report number of clusters and bootstrap replications. See Section~\ref{sec:dynamics-inference}.

\subsection*{Step 7: Report Dynamic Paths and Summaries}

\textbf{For event studies:}
\begin{itemize}
\item Report $\hat{\theta}_k$ for all lags with confidence intervals.
\item Plot event-time profile marking reference category.
\item Report $\widehat{\text{LRM}} = \hat{\theta}_{\bar{K}}$ (the effect at the longest lag, since event-time effects are cumulative) with CI.
\end{itemize}

\textbf{For distributed-lag models:}
\begin{itemize}
\item Report impulse response $\hat{\beta}_s$ with confidence intervals.
\item Report $\widehat{\text{LRM}} = \sum_s \hat{\beta}_s$.
\item Report half-life with CI.
\end{itemize}

Discuss substantive interpretation of dynamic profile, LRM, and half-life.

\subsection*{Step 8: Conduct Sensitivity Analyses}

\begin{itemize}
\item \textbf{Window:} Compare full vs restricted window.
\item \textbf{Binning:} Compare different bin thresholds.
\item \textbf{Lag length:} Compare $\bar{L} = 4, 6, 8, 10$.
\item \textbf{Functional form:} Compare geometric, polynomial, and flexible lags.
\item \textbf{Control group:} Compare never-treated vs not-yet-treated.
\end{itemize}

Report results for multiple specifications. Discuss most plausible based on diagnostics.

\subsection*{Step 9: Document Assumptions and Threats}

\textbf{State assumptions:} Parallel trends, no anticipation, overlap. Provide evidence (pre-trend tests, support plots, balance checks).

\textbf{Discuss threats:}
\begin{itemize}
\item Pre-trends (differential trends unrelated to treatment).
\item Anticipation (forward-looking behaviour).
\item Spillovers (contamination of control units).
\item Structural breaks (regime shifts at intervention).
\item Sparse support (unstable estimates at distant lags).
\end{itemize}

\textbf{Provide replication materials:} Data, scripts, software versions.

\begin{tcolorbox}[colback=green!5!white,colframe=green!50!black,title=Box 10.3: Dynamic Effects Checklist]
\textbf{1. Define Estimand:} Specify $\theta_k$, $\beta_s$, LRM, or half-life. Pre-specify window.

\textbf{2. Map Support:} Plot treated/control per event time. Flag sparse lags. Bin or restrict.

\textbf{3. Select Estimator:} Event study, distributed-lag, or dynamic DiD. Justify choice.

\textbf{4. Specify Leads/Lags:} Define windows. Specify reference category. Document rationale.

\textbf{5. Pre-Trend Checks:} Estimate leads. Plot with CIs. Joint F-test. Diagnose if rejected.

\textbf{6. Choose Inference:} Cluster-robust, two-way, HAC, or wild bootstrap. Document choice.

\textbf{7. Report Paths:} Plot event-time or impulse response. Report LRM and half-life with SE.

\textbf{8. Sensitivity:} Vary window, binning, lag length, functional form, control group.

\textbf{9. Document:} State assumptions. Discuss threats. Provide replication materials.
\end{tcolorbox}

\subsection*{Summary Tables and Figures}

\begin{table}[htbp]
\begin{tighttable}
\centering
\caption{Estimand–Estimator Mapping for Dynamic Treatment Effects}
\label{tab:dynamics-methods}
\begin{tabularx}{\textwidth}{Y Y Y Y}
\toprule
\textbf{Estimand} & \textbf{Data Structure} & \textbf{Estimator} & \textbf{Reference} \\
\midrule
Event-time effects $\theta_k$ & Binary treatment; staggered adoption & Heterogeneity-robust event study & Chapter~\ref{ch:did}, \ref{ch:event} \\
\addlinespace
Impulse responses $\beta_s$ & Continuous intensity; panel variation & Geometric ad-stock, Almon, flexible lags & This chapter \\
\addlinespace
Long-run multiplier & Binary or continuous; sufficient lags & $\theta_{\bar{K}}$ (event-time) or $\sum_s \beta_s$ (impulse); delta method & This chapter \\
\addlinespace
Half-life & Decaying effects; geometric structure & $\log 2 / (-\log \delta)$ & This chapter \\
\addlinespace
Dose-response $\mu(w, k)$ & Continuous; high-dimensional confounders & DML for panels & Chapter~\ref{ch:ml-nuisance}, \ref{ch:continuous} \\
\bottomrule
\end{tabularx}
\end{tighttable}
\end{table}

\begin{figure}[htbp]
\centering
\includegraphics[width=0.9\textwidth]{images/fig_dynamics_event.pdf}
\caption{Event-Time Path with Ramp-Up Pattern. The figure shows illustrative event-time effects $\hat{\theta}_k$ with 95\% confidence intervals. Pre-treatment leads ($k < -1$) are near zero, supporting parallel trends. Post-treatment effects show a gradual ramp-up, stabilising at approximately 10\% after three periods. The vertical dashed line marks the reference category ($k = -1$).}
\label{fig:dynamics-event}
\end{figure}

\begin{figure}[htbp]
\centering
\includegraphics[width=0.9\textwidth]{images/fig_dynamics_impulse.pdf}
\caption{Impulse Response under Geometric Ad-Stock. The figure shows the impulse response $\hat{\beta}_s = \hat{\beta}_0 \hat{\delta}^s$ with geometric decay ($\delta = 0.75$). The contemporaneous effect is $\beta_0 = 10\%$. The half-life (time to decay to $\beta_0/2$) is approximately 2.4 periods. The long-run multiplier (shaded area) is $\text{LRM} = \beta_0/(1-\delta) = 40\%$.}
\label{fig:dynamics-impulse}
\end{figure}
\index{dynamic treatment effects|)}
