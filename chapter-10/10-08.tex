\section{Marketing Applications}
\label{sec:dynamics-marketing}

Dynamic treatment effects are ubiquitous in marketing, where interventions generate time-varying responses through awareness-building, habit formation, stockpiling, and decay. This section illustrates dynamic methods through three hypothetical marketing applications. The numerical examples are illustrative and demonstrate how dynamic analysis would proceed in practice.

For estimation methods, see Section~\ref{sec:dynamics-estimation}. For diagnostics, see Section~\ref{sec:dynamics-diagnostics}. For inference, see Section~\ref{sec:dynamics-inference}.

\subsection*{Overview of Applications}

Table~\ref{tab:dynamics-applications} summarises the three applications.

\begin{table}[htbp]
\begin{tighttable}
\centering
\caption{Summary of Marketing Applications}
\label{tab:dynamics-applications}
\begin{tabularx}{\textwidth}{Y Y Y Y Y}
\toprule
\textbf{Application} & \textbf{Method} & \textbf{Data} & \textbf{Key Finding} & \textbf{Policy Implication} \\
\midrule
TV advertising & Distributed-lag (geometric ad-stock) & Weekly DMA sales & Carryover with decay & Pulsing vs continuous \\
\addlinespace
Promotions & Event study & Weekly store sales & Spike then dip pattern & Promotion spacing \\
\addlinespace
Loyalty programme & Dynamic DiD & Quarterly store revenue & Gradual ramp-up & ROI forecasting \\
\bottomrule
\end{tabularx}
\end{tighttable}
\end{table}

\subsection*{Application 1: Television Advertising with Carryover}

Television advertising provides a canonical setting for dynamic methods. The distributed-lag framework has a rich history in advertising response modelling \citet{tellis2000which,tellis2006optimal}. Meta-analytic evidence shows that advertising elasticities average 0.10 in the short run and 0.22 in the long run, with substantial heterogeneity \citet{sethuraman2011advertising}.

\paragraph{Setting.} Suppose a CPG brand launches a national TV campaign in 50 DMAs over four weeks, with varying GRPs across markets. Weekly sales are observed for 52 weeks before and 26 weeks after.

\paragraph{Goal.} Estimate the dynamic sales response; quantify half-life and LRM.

\paragraph{Method.} Geometric ad-stock model:
\[
\log(\text{Sales}_{it}) = \beta_0 \text{GRP}_{it} + \beta_1 A_{it} + \alpha_i + \gamma_t + \varepsilon_{it},
\]
where $A_{it} = \delta A_{i,t-1} + \text{GRP}_{i,t-1}$ is ad-stock with decay rate $\delta$.

\paragraph{Estimation.} NLS grid search over $\delta \in [0.5, 0.95]$.

\paragraph{Illustrative results.} Suppose the estimates yield $\hat{\beta}_0 = 0.03$ (3\% immediate lift per 100 GRPs), $\hat{\beta}_1 = 0.02$ (2\% lift per 100 units of ad-stock), and $\hat{\delta} = 0.8$ (20\% weekly decay). The implied half-life would be $\log 2 / (-\log 0.8) \approx 3.1$ weeks, and the LRM would be $(0.03 + 0.02/0.2) \times 100 = 13\%$ cumulative lift.

\paragraph{Diagnostics.} Residuals show no autocorrelation (Durbin-Watson). Geometric decay provides better fit than Almon polynomial (lower BIC). Cluster-robust SE clustered by DMA.

\paragraph{Policy.} The 3-week half-life implies pulsed schedules maintain awareness better than continuous. LRM informs budget allocation across markets.

\begin{tcolorbox}[colback=blue!5!white,colframe=blue!75!black,title=Box 10.1: Meta-Analytic Evidence on Advertising Elasticity]

Sethuraman, Tellis, and Briesch conduct a meta-analysis of 751 short-term and 402 long-term advertising elasticities from 56 studies spanning 1960--2008 \citet{sethuraman2011advertising}.

\textbf{Key findings:}
\begin{itemize}
\item Short-run elasticity averages 0.10; long-run averages 0.22.
\item Long-run/short-run ratio $\approx 2.2$ reflects carryover through awareness and habit.
\item Durables $>$ non-durables; TV $>$ print; new brands $>$ established brands.
\item Weekly data yield higher elasticities than monthly (aggregation bias).
\end{itemize}

\textbf{Implications:} Expect modest average effects (10\% elasticity means doubling ad spending increases sales by 10\%) but substantial heterogeneity. Context-specific estimation is essential.
\end{tcolorbox}

\subsection*{Application 2: Promotions with Post-Promotion Dips}

Multi-period promotions induce stockpiling, creating post-promotion dips that reduce net cumulative effect.

\paragraph{Setting.} Suppose a retailer offers 20\% discount every 4 weeks for 6 months (6 cycles) in 30 stores. Weekly sales are observed for 12 weeks before and 12 weeks after.

\paragraph{Goal.} Estimate dynamic sales response; quantify post-promotion dip; compute net cumulative effect.

\paragraph{Method.} Event-study specification:
\[
\log(\text{Sales}_{it}) = \sum_{k = -3}^{+5} \theta_k \mathds{1}\{t - g_{im} = k\} + \alpha_i + \gamma_t + \varepsilon_{it},
\]
where $g_{im}$ is the week of the $m$-th promotion, with $k = -1$ as reference.

\paragraph{Illustrative results.} Suppose the event-study estimates show $\hat{\theta}_0 = 0.35$ (42\% sales lift in promotion week), followed by $\hat{\theta}_1 = -0.15$ (14\% dip one week after) and $\hat{\theta}_2 = -0.10$ (10\% dip two weeks after), with return to baseline by week three. If pre-trends are near zero, the net cumulative effect would be $0.35 - 0.15 - 0.10 = 0.10$ (10\% net lift).

\paragraph{Diagnostics.} Support adequate (all 6 cycles contribute). Pre-trend test: F-test $p = 0.42$. Window sensitivity: results stable for $k \in [0, 8]$. Wild cluster bootstrap with 1,000 replications.

\paragraph{Policy.} In this hypothetical example, the post-promotion dip reduces the net effect to less than one-third of the immediate lift. Spacing promotions further apart would allow inventories to deplete and increase the net cumulative effect.

\subsection*{Application 3: Loyalty Programmes with Ramp-Up}

Loyalty programmes generate effects that build gradually as customers enrol, learn benefits, and develop habits.

\paragraph{Setting.} Suppose a retailer launches a loyalty programme in 50 stores over 6 quarters (staggered adoption, 10 stores per cohort). Quarterly revenue is observed for 8 quarters before and 8 quarters after.

\paragraph{Goal.} Estimate dynamic revenue response; quantify ramp-up period; assess persistence.

\paragraph{Method.} Dynamic DiD with heterogeneity-robust aggregation:
\[
\hat{\theta}_k = \frac{\sum_g N_g \widehat{\text{ATT}}(g, g+k)}{\sum_g N_g},
\]
where $\text{ATT}(g, t)$ compares cohort $g$ to never-treated stores.

\paragraph{Illustrative results.} Suppose the dynamic DiD estimates show a gradual ramp-up: $\hat{\theta}_0 = 0.02$ (2\% lift in first quarter), $\hat{\theta}_1 = 0.05$ (5\% in second quarter), $\hat{\theta}_2 = 0.08$ (8\% in third quarter), stabilising at approximately 10\% lift thereafter. If pre-trends are near zero, this pattern suggests a 3-quarter ramp-up period with persistent effects.

\paragraph{Diagnostics.} Support adequate for $k \in [-3, 8]$. Pre-trend test: F-test $p = 0.68$. Cohort weights: uniform vs size-weighted produce similar estimates (homogeneous effects). Cluster-robust SE clustered by store.

\paragraph{Policy.} In this hypothetical example, the 3-quarter ramp-up would inform ROI forecasting. Cumulative revenue lift compared to operating costs determines programme viability.

\subsection*{Vector Autoregressions for Marketing-Finance Links}

VARs trace dynamic feedback among marketing investments, consumer response, and firm value. Shocks to advertising or satisfaction propagate through sales and user-generated content to stock returns. Identification relies on timing restrictions or external instruments, and impulse responses summarise horizon-by-horizon returns to marketing assets \citet{srinivasan2009marketing,tirunillai2012chatter}.

\subsection*{Summary: Method Selection}

\begin{tcolorbox}[colback=green!5!white,colframe=green!50!black,title=Practical Guidance: Matching Method to Application]
\textbf{Continuous treatment with carryover (advertising):}
\begin{itemize}
\item Distributed-lag model with geometric ad-stock.
\item Estimate half-life, LRM.
\end{itemize}

\textbf{Discrete treatment with stockpiling (promotions):}
\begin{itemize}
\item Event-study design.
\item Visualise spike, dip, and net cumulative effect.
\end{itemize}

\textbf{Staggered adoption with ramp-up (loyalty programmes):}
\begin{itemize}
\item Dynamic DiD with heterogeneity-robust aggregation.
\item Estimate ramp-up period and persistence.
\end{itemize}

\textbf{Marketing-finance linkages:}
\begin{itemize}
\item VAR with impulse response functions.
\item Trace shocks through to stock returns.
\end{itemize}
\end{tcolorbox}
