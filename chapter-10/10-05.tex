\section{Anticipation, Carryover, and Mediation}
\label{sec:dynamics-anticipation}

Dynamic treatment effects arise from three distinct mechanisms: anticipation (forward-looking behaviour adjusting current outcomes in response to expected future treatments), carryover (lagged effects of past treatments persisting into current periods), and mediation (indirect effects operating through intermediate variables).

This section discusses how to diagnose and interpret each mechanism. For identification assumptions, see Section~\ref{sec:dynamics-identification}. For estimation, see Section~\ref{sec:dynamics-estimation}.

\subsection*{Overview of Dynamic Mechanisms}

Table~\ref{tab:dynamic-mechanisms} summarises the three mechanisms.

\begin{table}[htbp]
\begin{tighttable}
\centering
\caption{Comparison of Dynamic Mechanisms}
\label{tab:dynamic-mechanisms}
\begin{tabularx}{\textwidth}{Y Y Y Y}
\toprule
\textbf{Mechanism} & \textbf{Direction} & \textbf{Marketing Examples} & \textbf{Diagnostic} \\
\midrule
Anticipation & Future $\to$ Present & Delay buying before promotion; stockpile before price increase & Event-time leads ($k < 0$) \\
\addlinespace
Carryover & Past $\to$ Present & Advertising builds awareness; promotion generates trial & Event-time lags ($k > 0$) \\
\addlinespace
Mediation & Treatment $\to$ Mediator $\to$ Outcome & Advertising $\to$ Awareness $\to$ Sales & Path analysis; front-door \\
\bottomrule
\end{tabularx}
\end{tighttable}
\end{table}

\subsection*{Anticipation}

Anticipation occurs when agents are forward-looking and adjust current behaviour in response to expected future treatments. In marketing, consumers may delay buying in anticipation of a promotion (reducing current sales) or accelerate buying before a price increase (stockpiling). Search intensity often increases before a product launch, generating buzz. Firms may also adjust pricing, advertising, or inventory in anticipation of competitor actions.

Anticipatory responses bias event-time estimates if they occur in the pre-treatment period, because pre-treatment outcomes no longer reflect the untreated potential outcome but rather a mixture of the untreated outcome and the anticipatory response. This violates Assumption~\ref{ass:no-anticipation}.

\subsubsection*{Diagnosing Anticipation}

Estimate the event-study specification including leads ($k < 0$) alongside lags ($k \geq 0$). Plot $\hat{\theta}_k$ for $k < 0$ and test whether they are jointly zero using an F-test.

\begin{itemize}
\item \textbf{Leads near zero:} No anticipation is supported.
\item \textbf{Leads large:} Anticipation or pre-trends present.
\end{itemize}

\subsubsection*{Distinguishing Anticipation from Pre-Trends}

\begin{tcolorbox}[colback=orange!5!white,colframe=orange!50!black,title=Distinguishing Anticipation from Pre-Trends]
\textbf{Evidence for anticipation:}
\begin{itemize}
\item Effect begins exactly one period before treatment.
\item Effect is absent at earlier leads ($k = -2, -3, \ldots$).
\item Institutional knowledge confirms agents could anticipate (pre-announcement).
\end{itemize}

\textbf{Evidence for pre-trends:}
\begin{itemize}
\item Effect present at multiple leads.
\item Linear trend pattern across leads.
\item Agents could not have known about treatment.
\end{itemize}

\textbf{Design-based test:} Compare pre-announced vs surprise treatments. If the dip is present only for pre-announced treatments, anticipation is supported.
\end{tcolorbox}

\subsubsection*{Handling Anticipation}

\begin{tcolorbox}[colback=green!5!white,colframe=green!50!black,title=Practical Guidance: Options When Anticipation is Present]
\textbf{Option 1: Model anticipation explicitly.}
\begin{itemize}
\item Include leads in the specification and interpret them as anticipatory effects.
\item Report both anticipatory effects (leads) and post-treatment effects (lags).
\item Discuss total effect = anticipation + post-treatment.
\end{itemize}

\textbf{Option 2: Redefine intervention time.}
\begin{itemize}
\item If consumers anticipate one period in advance, redefine treatment to switch on at $g-1$.
\item Captures onset of anticipation as the treatment.
\end{itemize}

\textbf{Option 3: Exclude anticipation period.}
\begin{itemize}
\item Use only periods $t \leq g - \bar{a}$ as pre-treatment, where $\bar{a}$ is anticipation window.
\item Restricts comparison to periods before anticipation began.
\end{itemize}

\textbf{Choice depends on:} Substantive question; strength of evidence; whether anticipation is of independent interest.
\end{tcolorbox}

\subsection*{Carryover and Decay}

Carryover describes how treatment effects persist and dissipate over lags.

\paragraph{Positive carryover.} Past treatments can increase current outcomes. Advertising builds brand awareness as a cumulative stock. Promotions generate trial that leads to repeat buying. Loyalty programmes create habits and routines.

\paragraph{Negative carryover.} Past treatments can also decrease current outcomes. Promotions may exhaust demand through stockpiling. Repeated exposures can cause satiation and wear-out.
\begin{tcolorbox}[colback=blue!5!white,colframe=blue!75!black,title=Box 10.4: The Attention Paradox in Dynamic Advertising Effects]
\textbf{Setting.} A video platform runs a six-week campaign with a high-frequency pre-roll ad for a focal brand. Let $W_{it}$ indicate exposure to the high-frequency schedule in week $t$ for market $i$, and let $G_i$ be the week when the schedule is turned on. We estimate cohort-specific event-time effects $\theta_k(g)$ and aggregated effects $\theta_k$ as in Definition~\ref{def:event-time-effects}, with $k = t - G_i$.

\textbf{Dynamic pattern.} Suppose the estimated profile $\{\hat{\theta}_k\}$ for $k = 0,1,\ldots,8$ shows:
\begin{itemize}
\item $\hat{\theta}_0, \hat{\theta}_1, \hat{\theta}_2 > 0$: early weeks of high attention increase sales and brand search.
\item $\hat{\theta}_3 \approx 0$: additional exposure is neutral.
\item $\hat{\theta}_4, \hat{\theta}_5 < 0$: further weeks of the same creative reduce sales relative to a counterfactual with shorter duration.
\end{itemize}
The cumulative effect through horizon $H$ is $\text{LRM}(H) = \hat{\theta}_H$ if event-time effects are cumulative, or $\sum_{k=0}^H \hat{\beta}_k$ if working with incremental impulse responses $\hat{\beta}_k = \hat{\theta}_k - \hat{\theta}_{k-1}$. In this pattern $\text{LRM}(3)$ is positive and substantial, but $\text{LRM}(6)$ is only slightly larger or even smaller if the late negative effects dominate. More ``attention'' in the sense of longer exposure to the same ad generates negative carryover.

\textbf{Measurement implication.} Focusing only on early positive lags or on total impressions obscures the attention paradox: past a point, incremental event-time effects turn negative even as exposure accumulates. Reporting the full $\hat{\theta}_k$ trajectory, cumulative effects $\text{LRM}(H)$ for multiple horizons, and confidence bands helps identify where attention becomes counterproductive and informs optimal campaign duration and rotation of creatives.
\end{tcolorbox}

\subsubsection*{Interpreting Lag Profiles}

Interpreting lag profiles as carryover requires care. Elevated post-treatment sales could reflect true carryover (consumers who tried during promotion continue buying) or sample selection (promotion attracted high-valuation consumers who would have bought eventually).

Distinguishing these requires additional evidence (e.g., tracking same consumers over time; comparing exposed vs unexposed consumers).

\subsubsection*{Geometric Decay}

Geometric decay is the most common functional form:
\[
\beta_s = \beta_0 \delta^s,
\]
where $\beta_0$ is the immediate effect and $\delta \in (0, 1)$ is the decay rate.

The long-run multiplier is $\text{LRM} = \beta_0 / (1 - \delta)$, and the half-life is $h^* = \log 2 / (-\log \delta)$.

\paragraph{Sensitivity analysis.} Compare geometric, polynomial, and flexible lags to assess robustness to functional form.

\subsection*{Mediation}

Mediation channels describe pathways through which treatments affect outcomes.

\begin{table}[htbp]
\begin{tighttable}
\centering
\caption{Mediation Channels in Marketing}
\label{tab:mediation-channels}
\begin{tabularx}{\textwidth}{Y Y Y}
\toprule
\textbf{Treatment} & \textbf{Mediators} & \textbf{Outcome} \\
\midrule
Advertising & Awareness $\to$ Consideration $\to$ Trial & Sales \\
\addlinespace
Promotions & Deal-seeking $\to$ Stockpiling $\to$ Category expansion & Sales \\
\addlinespace
Loyalty programmes & Switching costs $\to$ Habit formation $\to$ Rewards & Retention, Sales \\
\bottomrule
\end{tabularx}
\end{tighttable}
\end{table}

Understanding mediation channels informs design (which channels to activate), targeting (which customers respond to which channels), and optimisation (how to allocate budget across channels).

\subsubsection*{Caution: Post-Treatment Conditioning}

\begin{tcolorbox}[colback=red!5!white,colframe=red!50!black,title=Warning: Post-Treatment Conditioning Bias]
\textbf{Problem:} Naïve mediation analysis conditions on post-treatment mediators (e.g., regressing sales on treatment and reward redemption). This:
\begin{itemize}
\item Opens backdoor paths through confounders of mediator-outcome.
\item Closes frontdoor paths that represent the causal effect.
\item Produces estimates without causal interpretation.
\end{itemize}

\textbf{Requirements for credible mediation:}
\begin{itemize}
\item Sequential ignorability (no unmeasured confounding of mediator-outcome).
\item Specialised methods: front-door adjustment, IV for mediator.
\end{itemize}

See \citet{pearl2009causality} for formal treatment.
\end{tcolorbox}

\subsubsection*{Practical Guidance for Mediation}

\begin{tcolorbox}[colback=green!5!white,colframe=green!50!black,title=Practical Guidance: Mediation in Marketing Panels]
\textbf{Report total effects as primary estimand:}
\begin{itemize}
\item Total effect = overall effect of treatment on outcome.
\item Clear causal interpretation; informs policy.
\end{itemize}

\textbf{Report path-specific effects only when:}
\begin{itemize}
\item Sequential ignorability is plausible.
\item Assumptions are explicitly discussed.
\item Sensitivity analyses assess robustness.
\end{itemize}

\textbf{Avoid conditioning on post-treatment variables} without explicit justification.

\textbf{When feasible, use experimental variation in mediators:}
\begin{itemize}
\item Randomise mediator (e.g., reward reminders).
\item Isolates specific channels with design-based credibility.
\end{itemize}
\end{tcolorbox}
